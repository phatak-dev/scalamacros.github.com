\documentclass[svgnames,dvipsnames,hyperref={bookmarks=false}]{beamer}
\useoutertheme{infolines}
\setbeamertemplate{headline}{} % removes the headline that infolines inserts
% \setbeamertemplate{footline}{
%   \hfill%
%   \usebeamercolor[fg]{page number in head/foot}%
%   \usebeamerfont{page number in head/foot}%
%   \insertpagenumber\,/\,\insertpresentationendpage\kern1em\vskip2pt%
% }
\setbeamertemplate{footline}{
  \hfill%
  \usebeamercolor[fg]{page number in head/foot}%
  \usebeamerfont{page number in head/foot}%
  \footnotesize\insertpagenumber\kern1em\vskip2pt%
}
\setbeamertemplate{navigation symbols}{}
\setbeamercolor{alerted text}{fg=blue}
\setbeamerfont{alerted text}{series=\bfseries,family=\ttfamily}
\setbeamertemplate{section in toc}{\text{\color{black}{\inserttocsection}}}
\usepackage[parfill]{parskip}
\usepackage[linewidth=0.5pt]{mdframed}
\newmdenv[innerleftmargin=1mm, innerrightmargin=1mm, innertopmargin=-1mm, innerbottommargin=2mm, leftmargin=-1mm, rightmargin=-1mm]{lstlistinglike}
\usepackage{tikz}
\usepackage{graphicx}
\DeclareGraphicsExtensions{.pdf,.png,.jpg}
\usepackage{textcomp}
\usepackage{pifont}
\usetikzlibrary{shapes.arrows}
\usepackage[T2A,T1]{fontenc}
\newcommand{\ZheZhe}{\mbox{\usefont{T2A}{\rmdefault}{m}{n}\CYRZH\CYRZH}}
\newcommand{\Meta}{\mbox{\usefont{T2A}{\rmdefault}{m}{n}\cyrm\cyre\cyrt\cyra\cyrp\cyrr\cyro\cyrg\cyrr\cyra\cyrm\cyrm\cyri\cyrr\cyro\cyrv\cyra\cyrn\cyri\cyre}}
\newcommand{\nextitem}{\par\hspace*{\labelsep}\textbullet\hspace*{\labelsep}}
\definecolor{light-gray}{gray}{0.80}

\tikzset{
    myarrow/.style={
        draw,
        fill=orange,
        single arrow,
        minimum height=3.5ex,
        single arrow head extend=1ex
    }
}
\newcommand{\arrowup}{%
\tikz [baseline=-0.5ex]{\node [myarrow,rotate=90] {};}
}
\newcommand{\arrowdown}{%
\tikz [baseline=-1ex]{\node [myarrow,rotate=-90] {};}
}

\tikzset{
    myshadedarrow/.style={
        draw,
        fill=light-gray,
        single arrow,
        minimum height=3.5ex,
        single arrow head extend=1ex
    }
}
\newcommand{\shadedarrowup}{%
\tikz [baseline=-0.5ex]{\node [myshadedarrow,rotate=90] {};}
}
\newcommand{\shadedarrowdown}{%
\tikz [baseline=-1ex]{\node [myshadedarrow,rotate=-90] {};}
}

\hypersetup{pdfauthor={Eugene Burmako},pdfsubject={Rethinking Scala Macros},pdftitle={Rethinking Scala Macros}}
\title{Rethinking Scala Macros}

\begin{document}

\title{Rethinking Scala Macros}
\subtitle{\textcolor{red}{Work in progress, not available yet}}
\author{Eugene Burmako}
\institute{\'Ecole Polytechnique F\'ed\'erale de Lausanne \\
           \texttt{http://scalamacros.org/}}
\date{02 March 2014}
{
\setbeamertemplate{footline}{}
\begin{frame}
  \titlepage
\end{frame}
}

\begin{frame}[fragile]
\frametitle{Outline}

\begin{itemize}
\item What is Palladium?
\item Planned features
\item Planned deliverables
\item Final words
\end{itemize}
\end{frame}

\begin{frame}[fragile]
\frametitle{}

\vskip40pt
\begin{center}
\text{\color{blue}{\Large{What is Palladium?}}}
\end{center}
\end{frame}

\begin{frame}[fragile]
\frametitle{Project Palladium}

\begin{itemize}
\item Successor of Project Kepler
\item Goal of Project Kepler: bring macros to Scala
\item Goal of Project Palladium: make macros in Scala easy to use
\end{itemize}
\end{frame}

\begin{frame}[fragile]
\frametitle{Scala macros: the good parts}

\begin{itemize}
\item Enable \text{\color{blue}\href{http://scalamacros.org/paperstalks/2014-02-04-WhatAreMacrosGoodFor.pdf}{cool use cases}} that were previously impossible/impractical
\item Have a significant community of research and production users
\item A lot of popular libraries in Scala ecosystem use macros
\end{itemize}
\end{frame}

\begin{frame}[fragile]
\frametitle{Scala macros: the bad parts}

\begin{itemize}
\item Using macros is easy, developing macros is hard
\item This contributes to the public image of metaprogramming
\item Useful, but hacky and obscure
\end{itemize}

\vskip25pt
\begin{quote}
I'm very envious of Racket macros, because it's very extensible. But I don't know how to do it for Haskell. TH is the nearest, but it's nowhere near.
\end{quote}
\begin{flushright}
\textemdash Simon Peyton Jones
\end{flushright}
\end{frame}

\begin{frame}[fragile]
\frametitle{Palladium goal \#1: Being straightforward}

\begin{semiverbatim}
coll.map(x => x + 1)

                          \arrowdown

\{
  def fn(x: Int) = x + 1
  val buf = Coll.newBuilder[T]
  var i = 0
  while (i < coll.length) buf += fn(coll(i))
  buf.result
\}

\end{semiverbatim}

\begin{itemize}
\item A canonical example that illustrates current problems with macros
\item Currently possible, but prohibitively complex to get right
\item To goal of Palladium is to make such macros writeable on autopilot
\end{itemize}
\end{frame}

\begin{frame}[fragile]
\frametitle{Palladium goal \#2: Being portable}

The trick is to make this work with:
\begin{itemize}
\item Scala compilers other than \texttt{scalac}
\item Integrated development environments
\item Incremental compilation
\item Interactive documentation
\item Runtime reflection
\end{itemize}
\end{frame}

\begin{frame}[fragile]
\frametitle{Summary}

Palladium will make macros straightforward and portable
\end{frame}

\begin{frame}[fragile]
\frametitle{}

\vskip40pt
\begin{center}
\text{\color{blue}{\Large{Planned features}}}
\end{center}
\end{frame}

\begin{frame}[fragile]
\frametitle{Our running example}

\begin{semiverbatim}
coll.map(x => x + 1)

                          \arrowdown

\{
  def fn(x: Int) = x + 1
  val buf = Coll.newBuilder[T]
  var i = 0
  while (i < coll.length) buf += fn(coll(i))
  buf.result
\}

\end{semiverbatim}

\begin{itemize}
\item Let's take another look at Paul's declosurify
\item Possible but ridiculously hard at the moment
\item How can Palladium help?
\end{itemize}
\end{frame}

\begin{frame}{Disclaimer}
\begin{itemize}
\item What follows is just a sketch, nothing's implemented yet
\item We might or might not be able to figure out everything
\item But all in all, the plan seems reasonable enough
\item After we have results, we'll see how/when this can be part of Scala
\end{itemize}
\end{frame}

\begin{frame}[fragile]
\frametitle<1>{Feature \#1: Simple definitions}
\frametitle<2>{Feature \#2: Simple reflection}
\frametitle<3>{Feature \#3: Simple trees}
\frametitle<4>{Feature \#4: Simple types}
\frametitle<5>{Feature \#5: Simple symbols}

\begin{semiverbatim}\small{
\alert<2>{import scala.reflect._}
import scala.language.macros

implicit class Mapper[Coll[_], A](coll: Coll[A]) \{
  \alert<1>{macro map[B](fn: A => B): Coll[B]} = \{
    val \alert<3>{q"(..\$ps) => \$body"} = fn
    val newBuilder = \alert<5>{\alert<2>{\alert<4>{t"Coll"}.companion.method("newBuilder")}}
    \alert<3>{q"""
      \alert<5>{def fn(..\$ps) = \$body}
      val buf = \$newBuilder[\alert<4>{\$A}]
      var i = 0
      while (i < coll.length) buf += fn(coll(i))
      buf.result
    """}
  \}
\}}
\end{semiverbatim}

\begin{itemize}
\only<1>{\item No longer necessary to split macro defs and macro impls}
\only<1>{\item No longer necessary to write tiresome c.Expr and c.WeakTypeTag}
\only<2>{\item Explicit macro context will be gone, along with path dependencies}
\only<2>{\item Redesigned reflection API that makes introspection and codegen easy}
\only<3>{\item No more manual construction/deconstruction, reification, exprs}
\only<3>{\item Trees won't carry types or symbols, but will be typecheckable}
\only<4>{\item Convenient notation to construct and deconstruct types}
\only<4>{\item No more tags, no more \texttt{case TypeRef(...)}, no more \texttt{appliedType}}
\only<5>{\item Symbols as we know them should be gone for good}
\only<5>{\item Introspection serviced by \texttt{Member}s, bindings handled by hygiene}
\end{itemize}
\end{frame}

\begin{frame}[fragile]
\frametitle{Feature \#6: Inline expansion}
\begin{itemize}
\item We can treat macro applications as folded regions of code
\item When you press [+], a given macro application expands
\item When you press [-], a given macro expansion collapses back
\end{itemize}
\end{frame}

\begin{frame}[fragile]
\frametitle{Feature \#7: Expansion error highlighting}
\begin{itemize}
\item Inline expansion will provide long-awaited interactivity
\item For one, errors in macro expansions are going to make sense
\item Have an error? Click [+] and see what exactly causes it!
\end{itemize}
\end{frame}

\begin{frame}[fragile]
\frametitle{Feature \#8: Expansion error troubleshooting}
\begin{itemize}
\item Quasiquotes can be smart, capturing locations they originate from
\item That would enable tracking culprits of errors in generated code
\item One could even imagine interactive fixes to codegen errors
\end{itemize}
\end{frame}

\begin{frame}[fragile]
\frametitle{Feature \#9: Inline debugging}
\begin{itemize}
\item The concept of interactive expansion is also applicable to debugging
\item Once a macro is expanded, you will be able to set breakpoints in expanded code
\end{itemize}
\end{frame}

\begin{frame}[fragile]
\frametitle{Feature \#10: Incremental compilation}

SBT will correctly handle macro expansions:
\begin{itemize}
\item No more whole project recompilations on a tiny change in a macro
\item Changes to macro arguments will recompile expansions
\item Changes to macro bodies and their helpers will recompile expansions
\item Changes to types introspected by macros will recompile expansions
\end{itemize}
\end{frame}

\begin{frame}[fragile]
\frametitle{Summary}

\begin{itemize}
\item Simple macro definitions
\item Simple reflection API
\item Interactive expansion
\item Inline debugging
\item Incremental compilation
\end{itemize}
\end{frame}

\begin{frame}[fragile]
\frametitle{}

\vskip40pt
\begin{center}
\text{\color{blue}{\Large{Planned deliverables}}}
\end{center}
\end{frame}

\begin{frame}[fragile]
\frametitle{M1}

\begin{itemize}
\item Aims to deliver a demoable prototype of the Palladium macro system
\item That works nicely with the existing ecosystem of tools
\item And is reasonably compatible with existing popular macros
\item By ScalaDays 2014 (16-18 June)
\end{itemize}
\end{frame}

\begin{frame}[fragile]
\frametitle{Component \#1: New reflection API}

\begin{itemize}
\item Reflection Core, a redesigned compile-time/runtime reflection library
\item Interface shared between Scala, Dotty, Eclipse, Intellij, SBT, etc
\item Specced and developed independently of implementors
\end{itemize}
\end{frame}

\begin{frame}[fragile]
\frametitle{Component \#2: Hygienic quasiquotes}

\begin{itemize}
\item Smart quasiquoting facility that respects hygiene and ref transparency
\item Very much relies on getting trees right
\item Denys will elaborate on that at Scala Days
\end{itemize}
\end{frame}

\begin{frame}[fragile]
\frametitle{Component \#3: AST interpretation}

\begin{itemize}
\item Macros will run in an interpreter, ensuring portability and compatibility
\item Having an AST interpreter is also useful beyond macro expansion
\item For example, it will give us a fast and minimalistic REPL!
\end{itemize}
\end{frame}

\begin{frame}[fragile]
\frametitle{Component \#4: AST persistence}

\begin{itemize}
\item In order to interpret macros, we need to store their ASTs
\item And not only their ASTs, but also ASTs of their dependencies
\item Ramping this up, how about we store ASTs for everything?!
\item AST persistence is also useful beyond macro expansion
\end{itemize}
\end{frame}

\begin{frame}[fragile]
\frametitle{Components \#3+4: Runtime expansion}

\begin{itemize}
\item AST interpretation and AST persistence work very well together
\item Interpreted ASTs => we don't need the compiler to run macros
\item Persistent ASTs => we don't need the compiler to setup environment
\item As a result, we will be able to expand macros at runtime!!
\end{itemize}
\end{frame}

\begin{frame}[fragile]
\frametitle{Component \#5: Tooling infrastructure (SBT)}

\begin{itemize}
\item At the moment, SBT doesn't know almost anything about macros
\item A) If macro body changes, we've got to recompile, but we don't
\item B) If macro data changes, we've got to recompile, but we don't
\item With ASTs and interpretation traces, we can do so much better!
\end{itemize}
\end{frame}

\begin{frame}[fragile]
\frametitle{Component \#5: Tooling infrastructure (IDE)}

\begin{itemize}
\item Not much can be done if macros are just arbitrary functions
\item However with interpretation we can easily control expansions
\item The model of [+]/[-] buttons for macro applications
\item Both for interactive editing and debugging
\end{itemize}
\end{frame}

\begin{frame}[fragile]
\frametitle{Summary}

\begin{itemize}
\item Straightforward reflection API decoupled from compiler internals
\item Hygienic quasiquotes which are essential for tree manipulations
\item AST interpreter
\item AST persistence
\item Tooling infrastructure: incremental compilation and IDEs
\end{itemize}
\end{frame}

\begin{frame}[fragile]
\frametitle{}

\vskip40pt
\begin{center}
\text{\color{blue}{\Large{Final words}}}
\end{center}
\end{frame}

\begin{frame}[fragile]
\frametitle{Status}

\begin{itemize}
\item Palladium was kicked off just two weeks ago
\item Most of the team is from EPFL with several external contributors
\item It is a research platform for new metaprogramming technologies
\item Targetting Scala and Dotty
\end{itemize}
\end{frame}

\begin{frame}[fragile]
\frametitle{Feedback}

\begin{itemize}
\item Your feedback and contributions are very much welcome
\item Mailing list: \text{\color{blue}\href{https://groups.google.com/forum/\#!forum/palladium-dev}{palladium-dev @ groups.google.com}}
\item Design documents: \text{\color{blue}\href{https://drive.google.com/?authuser=0\#folders/0Bxbd8B9L-XfmcE9tRFBXVjZtY0k}{Palladium Shared @ docs.google.com}}
\end{itemize}
\end{frame}

\begin{frame}[fragile]
\frametitle{Summary}

\begin{itemize}
\item Palladium will make macros straightforward and portable
\item New reflection + AST interpretation + AST persistence + tooling
\item Welcome to the future of Scala macros!
\end{itemize}
\end{frame}

\end{document}