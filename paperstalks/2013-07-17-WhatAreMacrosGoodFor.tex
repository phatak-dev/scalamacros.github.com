\documentclass[svgnames,hyperref={bookmarks=false}]{beamer}
\useoutertheme{infolines}
\setbeamertemplate{headline}{} % removes the headline that infolines inserts
% \setbeamertemplate{footline}{
%   \hfill%
%   \usebeamercolor[fg]{page number in head/foot}%
%   \usebeamerfont{page number in head/foot}%
%   \insertpagenumber\,/\,\insertpresentationendpage\kern1em\vskip2pt%
% }
\setbeamertemplate{footline}{
  \hfill%
  \usebeamercolor[fg]{page number in head/foot}%
  \usebeamerfont{page number in head/foot}%
  \footnotesize\insertpagenumber\kern1em\vskip2pt%
}
\setbeamertemplate{navigation symbols}{}
\setbeamercolor{alerted text}{fg=blue}
\setbeamerfont{alerted text}{series=\bfseries,family=\ttfamily}
\setbeamertemplate{section in toc}{\text{\color{black}{\inserttocsection}}}
\usepackage[parfill]{parskip}
\usepackage[linewidth=0.5pt]{mdframed}
\newmdenv[innerleftmargin=1mm, innerrightmargin=1mm, innertopmargin=-1mm, innerbottommargin=2mm, leftmargin=-1mm, rightmargin=-1mm]{lstlistinglike}
\usepackage{tikz}
\usepackage{graphicx}
\DeclareGraphicsExtensions{.pdf,.png,.jpg}
\usepackage{textcomp}
\usepackage{pifont}
\usetikzlibrary{shapes.arrows}

\tikzset{
    myarrow/.style={
        draw,
        fill=orange,
        single arrow,
        minimum height=3.5ex,
        single arrow head extend=1ex
    }
}
\newcommand{\arrowup}{%
\tikz [baseline=-0.5ex]{\node [myarrow,rotate=90] {};}
}
\newcommand{\arrowdown}{%
\tikz [baseline=-1ex]{\node [myarrow,rotate=-90] {};}
}

\hypersetup{pdfauthor={Eugene Burmako},pdfsubject={What Are Macros Good For?},pdftitle={What Are Macros Good For?}}
\title{What Are Macros Good For?}

\begin{document}

\title{What Are Macros Good For?}
\author{Eugene Burmako}
\institute{\'Ecole Polytechnique F\'ed\'erale de Lausanne \\
           \texttt{http://scalamacros.org/}}
\date{17 July 2013}

{
\setbeamertemplate{footline}{}
\begin{frame}
  \titlepage
  This talk has a newer revision that lives at \text{\color{blue}\href{http://scalamacros.org/paperstalks/2013-12-02-WhatAreMacrosGoodFor.pdf}{scalamacros.org/paperstalks/2014-02-04-WhatAreMacrosGoodFor.pdf}}
\end{frame}
}

\begin{frame}[fragile]
\frametitle{What are macros good for?}

\begin{itemize}
\item Code generation
\item Static checks
\item Domain-specific languages
\end{itemize}
\end{frame}

\begin{frame}[fragile]
\frametitle{}

\vskip40pt
\begin{center}
\text{\color{blue}{\Large{Macros 101}}}
\end{center}
\end{frame}

\begin{frame}[fragile]
\frametitle{What are macros?}

\begin{itemize}
\item An experimental feature of 2.10.x and 2.11.0
\item You write functions against the reflection API
\item Compiler invokes them during compilation
\end{itemize}
\end{frame}

\begin{frame}[fragile]
\frametitle{Macro flavors}

\begin{itemize}
\item Many ways to hook into the compiler \text{\textrightarrow} many macro flavors
\item Type macros, annotation macros, untyped macros, etc
\item However in 2.10.x and 2.11.0 there are only def macros
\end{itemize}
\end{frame}

\begin{frame}[fragile]
\frametitle{Def macros}

\begin{semiverbatim}
log(Error, "does not compute")

                          \arrowdown

if (Config.loggingEnabled)
  Config.logger.log(Error, "does not compute")

\end{semiverbatim}

\begin{itemize}
\item Def macros replace well-typed terms with other well-typed terms
\item Generated code might contain arbitrary Scala constructs
\item Codegeneration might involve arbitrary computations
\end{itemize}
\end{frame}

\begin{frame}[t, fragile]
\frametitle{Def macros}

\begin{semiverbatim}
def log(severity: Severity, msg: String): Unit\only<1>{ = ...







}\only<2->{ = macro impl}

\only<2->{def impl(c: Context)
    (severity: c.Expr[Severity],
     msg: c.Expr[String]): c.Expr[Unit]}\only<2>{ = ...





}\only<3->{ = \{
  import c.universe._
  reify \{
    if (Config.loggingEnabled)
      Config.logger.log(severity.splice, msg.splice)
  \}
\}}

\end{semiverbatim}

\begin{itemize}
\item Macro signatures look like signatures of normal methods
\only<2->{\item Macro bodies are just stubs, implementations are defined outside}
\only<3->{\item Implementations use reflection API to analyze and generate code}
\end{itemize}
\end{frame}

% \begin{frame}[t, fragile]
% \frametitle{Improvements in 2.11.0}

% \begin{semiverbatim}
% def log(severity: Severity, msg: String): Unit =
%   macro Log.impl

% trait Log extends Macro \{
%   import c.universe._
%   def impl(severity: c.Tree, msg: c.Tree): c.Tree = q"""
%     if (Config.loggingEnabled)
%       Config.logger.log(\$severity, \$msg)
%   """
% \}

% \end{semiverbatim}

% \begin{itemize}
% \item Quasiquotes can be used to construct and deconstruct trees
% \item Macros can be defined in bundles inheriting from \texttt{Macro}
% \item It's no longer mandatory to use \texttt{Expr}'s everywhere
% \end{itemize}
% \end{frame}

\begin{frame}[fragile]
\frametitle{Summary}

\begin{semiverbatim}
log(Error, "does not compute")

                          \arrowdown

if (Config.loggingEnabled)
  Config.logger.log(Error, "does not compute")

\end{semiverbatim}
\begin{itemize}
\item Local expansion of method calls
\item Well-formed and well-typed arguments
\item Now what is this good for?
\end{itemize}
\end{frame}

\begin{frame}[fragile]
\frametitle{}

\vskip40pt
\begin{center}
\text{\color{blue}{\Large{Code generation}}}
\end{center}
\end{frame}

\begin{frame}[fragile]
\frametitle{Code generation}

\begin{itemize}
\item Create code on-the-fly
\item More convenient and robust than textual codegen
\item Impossible to create globally visible classes
\end{itemize}
\end{frame}

\begin{frame}[fragile]
\frametitle{Example \#1 - Performance}

\begin{semiverbatim}
def createArray[T: ClassTag](size: Int, el: T) = \{
  val a = new Array[T](size)
  for (i <- 0 until size) a(i) = el
  a
\}

\end{semiverbatim}

\begin{itemize}
\item We want to write beautiful generic code, and Scala makes that easy
\item Unfortunately, abstractions oftentimes bring overhead
\item E.g. in this case erasure will cause boxing leading to a slowdown
\end{itemize}
\end{frame}

\begin{frame}[fragile]
\frametitle{Example \#1 - Performance}

\begin{semiverbatim}
def createArray[@specialized T: ClassTag](...) = \{
  val a = new Array[T](size)
  for (i <- 0 until size) a(i) = el
  a
\}

\end{semiverbatim}

\begin{itemize}
\item Methods can be \texttt{@specialized}, but it's viral and heavyweight
\item Viral = the entire call chain needs to be specialized
\item Heavyweight = specialization leads to duplication of bytecode
\end{itemize}
\end{frame}

\begin{frame}[fragile]
\frametitle{Example \#1 - Performance}

\begin{semiverbatim}
def createArray[T: ClassTag](size: Int, el: T) = \{
  val a = new Array[T](size)
  \alert{def specBody[@specialized T](el: T) \{}
    \alert{for (i <- 0 until size) a(i) = el}
  \alert{\}}
  \alert{classTag[T] match \{}
    \alert{case ClassTag.Int => specBody(el.asInstanceOf[Int])}
    \alert{...}
  \alert{\}}
  a
\}

\end{semiverbatim}

\begin{itemize}
\item We want to specialize just as much as we need
\item As described in the fresh \text{\color{blue}\href{http://lampwww.epfl.ch/\~hmiller/scala2013/resources/pdfs/paper10.pdf}{Bridging Islands of Specialized Code}} paper
\item But that's tiresome to do by hand, and here's where macros shine
\end{itemize}
\end{frame}

\begin{frame}[fragile]
\frametitle{Example \#1 - Performance}

\begin{semiverbatim}
\alert{def specialized[T: ClassTag](code: => T) = macro ...}

def createArray[T: ClassTag](size: Int, el: T) = \{
  val a = new Array[T](size)
  \alert{specialized[T] \{}
    for (i <- 0 until size) a(i) = el
  \alert{\}}
  a
\}
\end{semiverbatim}

\begin{itemize}
\item \texttt{specialized} macro gets pretty code and transforms it into fast code
\item This is a typical scenario of using macros for performance
\item \text{\color{blue}\href{https://issues.scala-lang.org/browse/SI-6711}{Can be non-trivial to implement}}, but things should become better
\end{itemize}
\end{frame}

\begin{frame}[fragile]
\frametitle{Example \#2 - Materialization}

\begin{semiverbatim}
trait Reads[T] \{
  def reads(json: JsValue): JsResult[T]
\}

object Json \{
  def fromJson[T](json: JsValue)
    (implicit fjs: Reads[T]): JsResult[T]
\}

\end{semiverbatim}

\begin{itemize}
\item Type classes are an idiomatic way of writing extensible code in Scala
\item This is an example of typeclass-based design in Play
\end{itemize}
\end{frame}

\begin{frame}[fragile]
\frametitle{Example \#2 - Materialization}

\begin{semiverbatim}
def fromJson[T](json: JsValue)
  (implicit fjs: Reads[T]): JsResult[T]

implicit val IntReads = new Reads[Int] \{
  def reads(json: JsValue): JsResult[T] = ...
\}

fromJson[Int](json) // you write
fromJson[Int](json)(IntReads) // you get

\end{semiverbatim}

\begin{itemize}
\item With type classes we externalize the moving parts
\item Instances of type classes are provided once
\item And then \texttt{scalac} fills them in automatically
\end{itemize}
\end{frame}

\begin{frame}[fragile]
\frametitle{Example \#2 - Before macros}

\begin{semiverbatim}
case class Person(name: String, age: Int)

implicit val personReads = (
  (__ \\ 'name).reads[String] and
  (__ \\ 'age).reads[Int]
)(Person)

\end{semiverbatim}

\begin{itemize}
\item Everything is done manually, hence boilerplate
\item There are alternatives, but they have downsides
\end{itemize}
\end{frame}

\begin{frame}[fragile]
\frametitle{Example \#2 - Vanilla macros (2.10.0)}

\begin{semiverbatim}
implicit val personReads = Json\alert{.reads[}Person\alert{]}

\end{semiverbatim}

\begin{itemize}
\item Boilerplate can be generated by a macro
\item The code ends up being the same as if it were written manually
\item Therefore performance remains excellent
\end{itemize}
\end{frame}

\begin{frame}[fragile]
\frametitle{Example \#2 - Implicit macros (2.10.2+)}

\begin{semiverbatim}
// no code necessary

\end{semiverbatim}

\begin{itemize}
\item Implicit values can be generated by a macro
\item Used with great success in \text{\color{blue}\href{http://github.com/scala/pickling}{scala-pickling}}
\end{itemize}
\end{frame}

\begin{frame}[fragile]
\frametitle{Example \#2 - Implicit macros (2.10.2+)}

\begin{semiverbatim}
trait Reads[T] \{ def reads(json: JsValue): JsResult[T] \}

object Reads \{
  \alert{implicit def materializeReads[T]: Reads[T] = macro ...}
\}

\end{semiverbatim}

\begin{itemize}
\item When \texttt{scalac} looks for implicits, it traverses the implicit scope
\item Implicit scope transcends lexical scope
\item Among others it includes members of the target’s companion
\end{itemize}
\end{frame}

\begin{frame}[fragile]
\frametitle{Example \#2 - Implicit macros (2.10.2+)}

\begin{semiverbatim}
fromJson[Person](json)

                          \arrowdown

fromJson[Person](json)(\alert{materializeReads[}Person\alert{]})

                          \arrowdown

fromJson[Person](json)(new Reads[Person]\{ ... \})

\end{semiverbatim}

\begin{itemize}
\item Every time a \texttt{Reads[T]} isn't found, the compiler will call our macro
\item More information in \text{\color{blue}\href{http://www.meetup.com/Bay-Area-Scala-Enthusiasts/events/121848382/}{my talk about materialization}}
\end{itemize}
\end{frame}

\begin{frame}[fragile]
\frametitle{Example \#2 - Implicit macros (2.10.2+)}

\begin{semiverbatim}
implicit def listShowable[T](implicit s: Showable[T]) =
  new Showable[List[T]] \{
    def show(x: List[T]) = \{
      x.map(s.show).mkString("List(", ", ", ")")
  \}
\}

show(List(42))
// prints: List(42)

\end{semiverbatim}

\begin{itemize}
\item This is an example of how macros synergize with other features
\item Here a macro is transparently integrated with a vanilla \texttt{listShowable}
\item We will see a couple more applications of the same principle today
\end{itemize}
\end{frame}

\begin{frame}[fragile]
\frametitle{Example \#2 - A \#typelevel alternative}

\begin{semiverbatim}
product   :: C[A] => C[B] => C[(A, B)]
coproduct :: C[A] => C[B] => C[Either[A, B]]
project   :: C[B] => (A => B, B => A) => C[A]

\end{semiverbatim}

\begin{itemize}
\item Lars Hupel \text{\color{blue}\href{https://speakerdeck.com/larsrh/generating-type-class-instances-automatically}{recently introduced}} an approach inspired by Haskell's GP
\item Uniform representation + type class combinators = materialization
\item Beautiful, but has some problems with performance
\end{itemize}
\end{frame}

\begin{frame}[fragile]
\frametitle{Example \#3 - Fake type providers}

\begin{semiverbatim}
println(Db.Coffees.all)
Db.Coffees.insert("Brazilian", 99, 0)

\end{semiverbatim}

\begin{itemize}
\item In F\# one can generate wrappers over datasources
\item These wrappers can then be used in a strongly-typed manner
\item Can this be implemented with def macros?
\end{itemize}
\end{frame}

\begin{frame}[fragile, t]
\frametitle{Example \#3 - Fake type providers}

\begin{semiverbatim}
\alert{def h2db(connString: String): Any = macro ...}
val db = \alert{h2db(}"jdbc:h2:coffees.h2.db"\alert{)}

                          \arrowdown

val db = \{
  trait Db \{
    case class Coffee(...)
    val Coffees: Table[Coffee] = ...
  \}
  new Db \{\}
\}
\end{semiverbatim}

\begin{itemize}
\item Unfortunately for this use case, def macros expand locally
\item Therefore the best we can have is a bunch of local classes
\end{itemize}
\end{frame}

\begin{frame}[fragile, t]
\frametitle{Example \#3 - Fake type providers}

\begin{semiverbatim}
scala> val db = h2db("jdbc:h2:coffees.h2.db")
db: AnyRef \{
  type Coffee \{ val name: String; val price: Int; ... \}
  val Coffees: Table[this.Coffee]
\} = \$anon\$1...

scala> db.Coffees.all
res1: List[Db\$1.this.Coffee] = List(Coffee(Brazilian,99,0))

\end{semiverbatim}

\begin{itemize}
\item Luckily for us, local classes are erased to structural types
% \item \text{\color{blue}\href{http://meta.plasm.us/posts/2013/06/19/macro-supported-dsls-for-schema-bindings/}{A little nudge suggested by Travis Brown}} makes \texttt{scalac} happy
\item This means that we are strongly-typed (static checking, completions)
\item But we have to live with reflective overhead of structural types
\end{itemize}
\end{frame}

\begin{frame}[fragile, t]
\frametitle{Example \#3 - Fake type providers}

\begin{semiverbatim}
class Coffee(row: Row[\alert{"...".type}]) with Dynamic \{
  \alert{def selectDynamic = macro ...}
\}

db: AnyRef\{type Coffee <: Dynamic; ...\}
coffee\alert{.}name // rewritten to: coffee.\alert{selectDynamic(}"name"\alert{)}

                          \arrowdown

coffee.row["name"].asInstanceOf[String]

\end{semiverbatim}

\begin{itemize}
\item It turns out that \text{\color{blue}\href{http://meta.plasm.us/posts/2013/07/11/fake-type-providers-part-2/}{we don't need to go through structural types}}
\item \texttt{Dynamic} typing + compile-time macros = static typing
\item But now we lost IDEs, because we no longer expose any members
\end{itemize}
\end{frame}

\begin{frame}[fragile, t]
\frametitle{Example \#3 - Fake type providers}

\begin{semiverbatim}
class Coffee(row: Row["...".type]) \{
  \alert{def name: String = macro ...}
\}

db: AnyRef\{type Coffee <: AnyRef\{\alert{def name: String}\}; ...\}
coffee.\alert{name}

                          \arrowdown

coffee.row["name"].asInstanceOf[String]

\end{semiverbatim}

\begin{itemize}
\item Therefore we need to go even deeper
\item Reflective calls + compile-time macros = static calls
\item Meet structural types powered by \text{\color{blue}\href{http://meta.plasm.us/posts/2013/07/12/vampire-methods-for-structural-types/}{vampire methods}}
\end{itemize}
\end{frame}

\begin{frame}[fragile]
\frametitle{Example \#3 - Real type providers}

\begin{semiverbatim}
\alert{@H2Db(}"jdbc:h2:coffees.h2.db"\alert{)} object Db
println(Db.Coffees.all)
Db.Coffees.insert("Brazilian", 99, 0)

\end{semiverbatim}

\begin{itemize}
\item This was a fun exercise stretching the boundaries of macrology
\item The technique was discovered just last weekend, so use it with care
\item In the meanwhile keep an eye on \text{\color{blue}\href{http://docs.scala-lang.org/overviews/macros/annotations.html}{macro annotations}}
\end{itemize}
\end{frame}

\begin{frame}[fragile]
\frametitle{}

\vskip40pt
\begin{center}
\text{\color{blue}{\Large{Static checks}}}
\end{center}
\end{frame}

\begin{frame}[fragile]
\frametitle{Static checks}

\begin{itemize}
\item Check your program during compilation
\item Report errors and warnings
\item Impossible to do global checks
\end{itemize}
\end{frame}

\begin{frame}[fragile]
\frametitle{Example \#4 - Strongly-typed strings}

\begin{semiverbatim}
scala> val x = "42"
x: String = 42

scala> "%d".format(x)
j.u.IllegalFormatConversionException: d != java.lang.String
  at java.util.Formatter\$FormatSpecifier.failConversion...
\only<2>{
scala> f"\$x\%d"
<console>:31: error: type mismatch;
 found   : String
 required: Int
}
\end{semiverbatim}

\begin{itemize}
\item Strings are typically perceived to be unsafe
\only<2>{\item Though with macros they don't have to be}
\only<2>{\item Even more so with the new string interpolation}
\end{itemize}
\end{frame}

\begin{frame}[fragile]
\frametitle{Example \#4 - Strongly-typed strings}

\begin{semiverbatim}
implicit class Formatter(c: StringContext) \{
  \alert{def f(args: Any*): String = macro ???}
\}

val x = "42"
f"\$x\%d" // rewritten into: StringContext("", "%d").\alert{f(}x\alert{)}

                          \arrowdown

\{
  val arg\$1: Int = x \alert{// doesn't compile}
  "%d".format(arg\$1)
\}

\end{semiverbatim}

\begin{itemize}
\item \texttt{f} is a macro that inserts type ascriptions in strategic places
\item Similar techniques can be used for regexen, binary literals, etc
\end{itemize}
\end{frame}

\begin{frame}[fragile]
\frametitle{Example \#4 - Quasiquotes}

\begin{semiverbatim}
reify(List[T](element.splice))

                          \arrowdown

q"List[\$T](\$element)"

\end{semiverbatim}

\begin{itemize}
\item Now our strings are both flexible and statically checked
\item This means that we can \text{\color{blue}\href{https://github.com/densh/talks/raw/master/ecoop-2013-07-02/index.pdf}{robustly embed}} entire languages
\item That's how quasiquotes were born (already in 2.11.0-M4)
\end{itemize}
\end{frame}

\begin{frame}[fragile, t]
\frametitle{Example \#5 - Akka typed channels}

\begin{semiverbatim}
trait Request
case class Command(msg: String) extends Request

trait Reply
case object CommandSuccess extends Reply
case class CommandFailure(msg: String) extends Reply

\only<1>{val actor = someActor}\only<2>{type Spec = (Request, Reply) :+: TNil}
\only<1>{actor ! Command}\only<2>{val actor = new ChannelRef[Spec](someActor)}
\alert{\only<2>{actor <-!- Command // doesn't compile}}

\end{semiverbatim}

\begin{itemize}
\only<1>{\item Akka actors are dynamically typed, i.e. the \texttt{!} method takes \texttt{Any}}
\only<1>{\item This loosens type guarantees provided by Scala}
\only<2>{\item We can implement type specification for actors even in standard Scala}
\only<2>{\item But this became practical only when we got macros}
\only<2>{\item \text{\color{blue}\href{http://doc.akka.io/docs/akka/snapshot/scala/typed-channels.html}{Akka typed channels}} are specifically designed to make use of macros}
\end{itemize}
\end{frame}

\begin{frame}[fragile]
\frametitle{Example \#5 - Akka typed channels}

\begin{semiverbatim}
type Spec = (Request, Reply) :+: TNil
val actor = new ChannelRef[Spec](someActor)
\alert{actor <-!- Command // doesn't compile}

\end{semiverbatim}

\begin{itemize}
\item The \texttt{<-!-} macro takes the type of its prefix and extracts the spec
\item Then it takes the argument type and validates it against the spec
\item This all can be done with implicits and type-level computations
\item But it will be non-trivial both for the programmer and for the users
\end{itemize}
\end{frame}

% \begin{frame}[fragile]
% \frametitle{Example \#5 - Akka typed channels}

% \begin{semiverbatim}
% def replyChannels(l: Type, msg: Type): List[Type] = \{
%   def rec(l: Type, acc: List[Type]): List[Type] = \{
%     l match \{
%       case TypeRef(_, _, TypeRef(_, _, in :: out :: Nil) :: tl :: Nil)
%       if msg <:< in =>
%           rec(tl, if (acc contains out) acc else out :: acc)
%       case TypeRef(_, _, _ :: tl :: Nil) => rec(tl, acc)
%       case _ => acc.reverse
%     \}
%   \}
%   val n = typeOf[Nothing]
%   if (msg =:= n) List(n) else rec(list, Nil)
% \}
% \end{semiverbatim}
% \end{frame}

\begin{frame}[fragile, t]
\frametitle{Example \#6 - Spores}

\begin{semiverbatim}
def future[T](body: => T) = ...

def receive = \{
  case Request(data) =>
    future \{
      val result = transform(data)
      sender ! Response(result)
    \}
\}

\end{semiverbatim}

\begin{itemize}
\item Capturing \texttt{sender} in the above closure is dangerous
\item That's because \texttt{sender} is not a value, but a stateful method
\item To validate captures we can use macros: \text{\color{blue}\href{http://docs.scala-lang.org/sips/pending/spores.html}{SIP-21 \text{\textendash} Spores}}
\end{itemize}
\end{frame}

\begin{frame}[fragile, t]
\frametitle{Example \#6 - Spores}

\begin{semiverbatim}
def future[T](body: \alert{Spore[T]}) = ...

\alert{def spore[T](body: => T): Spore[T] = macro ...}

def receive = \{
  case Request(data) =>
    future(\alert{spore \{}
      val result = transform(data)
      sender ! Response(result) \alert{// doesn't compile}
    \alert{\}})
\}

\end{semiverbatim}

\begin{itemize}
\item The \texttt{spore} macro takes its body and figures out all free variables
\item If any of the free variables are deemed dangerous, an error is reported
\end{itemize}
\end{frame}

\begin{frame}[fragile, t]
\frametitle{Example \#6 - Spores}

\begin{semiverbatim}
def future[T](body: Spore[T]) = ...

\alert{implicit def anyToSpore[T](body: => T): Spore[T] = macro ...}

def receive = \{
  case Request(data) =>
    future \alert{\{}
      val result = transform(data)
      sender ! Response(result) \alert{// doesn't compile}
    \alert{\}}
\}

\end{semiverbatim}

\begin{itemize}
\item The conversion to \texttt{Spore} can be made implicit
\item That will verify closures without bothering the user
\end{itemize}
\end{frame}

\begin{frame}[fragile]
\frametitle{}

\vskip40pt
\begin{center}
\text{\color{blue}{\Large{Domain-specific languages}}}
\end{center}
\end{frame}

\begin{frame}[fragile]
\frametitle{Domain-specific languages}

\begin{itemize}
\item Take a program written in an internal or external DSL
\item Work with it as with a domain-specific data structure
\end{itemize}
\end{frame}

\begin{frame}[fragile, t]
\frametitle{Example \#7 - Language-INtegrated Query}

\begin{semiverbatim}
val usersMatching = query[String, (Int, String)](
  "select id, name from users where name = ?")
usersMatching("John")

\only<2->{case class User(id: Column[Int], name: Column[String])}
\only<2->{users.filter(_.name === "John")}

\only<3->{case class User(id: Int, name: String)}
\only<3->{users\alert{.filter(}_.name == "John"\alert{)}}

\end{semiverbatim}

\begin{itemize}
\item Database queries can be written in SQL
\only<2->{\item They can also be written in a DSL, at times slightly awkward}
\only<3->{\item Or they can be written in Scala and virtualized by a macro}
\end{itemize}
\end{frame}

\begin{frame}[fragile]
\frametitle{Example \#7 - Language-INtegrated Query}

\begin{semiverbatim}
trait Query[T] \{
  \alert{def filter(p: T => Boolean): Query[T] = macro ...}
\}

val users: Query[User] = ...
users\alert{.filter(}_.name == "John"\alert{)}


                          \arrowdown

Query(Filter(users, Equals(Ref("name"), Literal("John"))))

\end{semiverbatim}

\begin{itemize}
\item The \texttt{filter} macro turns a method call on a query into another query
\item The derived query captures the previous one and the predicate
\item We've built a deeply-embedded query DSL
\end{itemize}
\end{frame}

\begin{frame}[fragile]
\frametitle{Example \#7 - Comparison with staging}

\begin{semiverbatim}
implicit def queryOps[T](q: Rep[Query[T]]) = new \{
  def filter(p: Rep[T] => Rep[Boolean]): Rep[Query[T]] = ...
\}

val users: Rep[Query[User]] = ...
users.filter(_.name == "John")

\end{semiverbatim}

\begin{itemize}
\item In order to deeply embed your DSL, work with \texttt{Rep[T]} instead of \texttt{T}
\item Powered by implicits and \texttt{scala-virtualized}, a fork of \texttt{scalac}
\item Staged execution: compile your program, stage it, run the result
\item This technique is called \text{\color{blue}\href{http://scala-lms.github.io/}{lightweight modular staging}}
\end{itemize}
\end{frame}

\begin{frame}[fragile]
\frametitle{Example \#7 - Comparison with staging}

\begin{itemize}
\item Macros allow for earlier error detection
\item Macros don't need stage annotations
\item Staging composes better
\end{itemize}
\end{frame}

\begin{frame}[fragile]
\frametitle{Example \#7 - Comparison with staging}

\begin{semiverbatim}
case class Coffee(name: String, price: Double)
val coffees: Queryable[Coffee] = Db.coffees

// closed world
coffees.\alert{filter(}c => c.price < 10\alert{)}

// open world
def isAffordable(c: Coffee) = c.price < 10
scoffees.\alert{filter(}isAffordable{)}

\end{semiverbatim}

\begin{itemize}
\item Code is only processed specially within macro arguments
\item Therefore separate compilation doesn't work out of the box
\item We experiment with ways of dealing with that
\end{itemize}
\end{frame}

\begin{frame}[fragile, t]
\frametitle{Example \#8 - Async}

\begin{semiverbatim}
val futureDOY: Future[Response] =
  WS.url("http://api.day-of-year/today").get

val futureDaysLeft: Future[Response] =
  WS.url("http://api.days-left/today").get

futureDOY.flatMap \{ doyResponse =>
  val dayOfYear = doyResponse.body
  futureDaysLeft.map \{ daysLeftResponse =>
    val daysLeft = daysLeftResponse.body
    Ok(s"\$dayOfYear: \$daysLeft days left!")
  \}
\}
\end{semiverbatim}

\begin{itemize}
\item Turning a synchronous program into an async one isn't easy
\item One has to manually manage callbacks, introduce temps, etc
\end{itemize}
\end{frame}

\begin{frame}[fragile, t]
\frametitle{Example \#8 - Async}

\begin{semiverbatim}
\alert{def async[T](body: => T): Future[T] = macro ...}
\alert{def await[T](future: Future[T]): T = macro ...}

\alert{async \{}
  val dayOfYear = \alert{await(}futureDOY\alert{)}.body
  val daysLeft = \alert{await(}futureDaysLeft\alert{)}.body
  Ok(s"\$dayOfYear: \$daysLeft days left!")
\alert{\}}



\end{semiverbatim}

\begin{itemize}
\item Turning a synchronous program into an async one isn't easy
\item Macros can do the transformation automatically: \text{\color{blue}\href{http://docs.scala-lang.org/sips/pending/async.html}{SIP-22 \text{\textendash} Async}}
\item C\# \texttt{yield} and Python generators can also be emulated by macros
\end{itemize}
\end{frame}

\begin{frame}[fragile, t]
\frametitle{Example \#9 - Datomisca}
\begin{semiverbatim}
scala> \alert{Query(}"""
|      [ :find ?e ?n
|        :in \$ ?char
|        :where  [ ?e :person/name ?n ]
|                [ ?e person/character ?char ]
|      ]
|      """\alert{)}
res0: TypedQueryAuto2[DatomicData, DatomicData, (DatomicData,
DatomicData)] = [ :find ?e ?n :in \$ ?char :where ... ]

\end{semiverbatim}

\begin{itemize}
\item Macros can also deeply embed external DSLs
\item In this example a query to Datomic becomes \text{\color{blue}\href{http://pellucidanalytics.github.io/datomisca/index.html}{first-class Scala citizen}}
\item This means static validation, typechecking and maybe even interop
\end{itemize}
\end{frame}

% \begin{frame}[fragile, t]
% \frametitle{Example \#9 - Quasiquotes}
% \begin{semiverbatim}
% @quasiquote object lambda \{
%   sealed abstract class Tree
%   case class Abs(v: Var, body: Tree) extends Tree
%   case class App(f: Tree, arg: Tree) extends Tree
%   case class Var(name: String) extends Tree

%   object parse extends StdTokenParsers \{
%     lexical.delimiters ++= List("(", ")", "\\", ".")
%     def main = rep1(parens | varr | abs | hole) ^^ App
%     def abs = ("\\\\" ~> (varr | hole) <~ ".") ~ main) ^^ Abs
%     def varr = ident ^^ Var
%     def parens = "(" ~> main <~ ")"
%   \}
% \}
% \end{semiverbatim}

% \begin{itemize}
% \item
% \item
% \end{itemize}
% \end{frame}

\begin{frame}[fragile]
\frametitle{}

\vskip40pt
\begin{center}
\text{\color{blue}{\Large{Summary}}}
\end{center}
\end{frame}

\begin{frame}[fragile]
\frametitle{What are macros good for?}

\begin{itemize}
\item Code generation
\item Static checks
\item Domain-specific languages
\end{itemize}
\end{frame}

\end{document}
