\documentclass[svgnames,hyperref={bookmarks=false}]{beamer}
\useoutertheme{infolines}
\setbeamertemplate{headline}{} % removes the headline that infolines inserts
% \setbeamertemplate{footline}{
%   \hfill%
%   \usebeamercolor[fg]{page number in head/foot}%
%   \usebeamerfont{page number in head/foot}%
%   \insertpagenumber\,/\,\insertpresentationendpage\kern1em\vskip2pt%
% }
\setbeamertemplate{footline}{
  \hfill%
  \usebeamercolor[fg]{page number in head/foot}%
  \usebeamerfont{page number in head/foot}%
  \footnotesize\insertpagenumber\kern1em\vskip2pt%
}
\setbeamertemplate{navigation symbols}{}
\setbeamercolor{alerted text}{fg=blue}
\setbeamerfont{alerted text}{series=\bfseries,family=\ttfamily}
\setbeamertemplate{section in toc}{\text{\color{black}{\inserttocsection}}}
\usepackage[parfill]{parskip}
\usepackage[linewidth=0.5pt]{mdframed}
\newmdenv[innerleftmargin=1mm, innerrightmargin=1mm, innertopmargin=-1mm, innerbottommargin=2mm, leftmargin=-1mm, rightmargin=-1mm]{lstlistinglike}
\usepackage{tikz}
\usepackage{graphicx}
\DeclareGraphicsExtensions{.pdf,.png,.jpg}
\usepackage{textcomp}
\usepackage{pifont}

\hypersetup{pdfauthor={Eugene Burmako},pdfsubject={Macro Paradise},pdftitle={Macro Paradise}}
\title{Macro Paradise}

\begin{document}

\title{Macro Paradise}
\author{Eugene Burmako}
\institute{\'Ecole Polytechnique F\'ed\'erale de Lausanne \\
           \texttt{http://scalamacros.org/}}
\date{18 December 2012}
{
\setbeamertemplate{footline}{}
\begin{frame}
  \titlepage
\end{frame}
}

\begin{frame}[fragile]
\frametitle{Before we begin}

I'd like to heartily thank:
\begin{itemize}
\item Early adopters and contributors fearlessly trying out macros and reflection since December 2011
\item Reflection group turning impossible problems into great ideas, one meeting at a time
\end{itemize}
\end{frame}

\begin{frame}[fragile]
\frametitle{Today's talk is about}

\begin{itemize}
\item Macros in upcoming Scala 2.10.0-final
\item New features implemented since RC1
\item Plans for the future
\end{itemize}

\vskip40pt

Screencast: \text{\color{blue}\href{http://vimeo.com/user8565009/macro-paradise-talk}{http://vimeo.com/user8565009/macro-paradise-talk}}

Discussion: \text{\color{blue}\href{http://groups.google.com/group/scala-language/browse\_thread/thread/21c0cdce38715771}{scala-language/thread/21c0cdce38715771}}
\end{frame}

\begin{frame}[fragile]
\frametitle{}

\vskip40pt
\begin{center}
\text{\color{blue}{\Large{State of the art}}}
\end{center}
\end{frame}

\begin{frame}[fragile]
\frametitle{Macros}

\begin{quote}
 Macros are functions that are called by the compiler during compilation. Within these functions the programmer has access to compiler APIs. \end{quote}

\begin{flushright}
\textemdash \text{\color{blue}\href{http://scalamacros.org}{ http://scalamacros.org}}
\end{flushright}
\end{frame}

\begin{frame}[fragile]
\frametitle{Def macros}
\begin{semiverbatim}
object Asserts \{
  def assertionsEnabled = ...
  def raise(msg: Any) = throw new AssertionError(msg)

  def assert(cond: Boolean, msg: Any) = macro impl
  def impl(c: Context)
          (cond: c.Expr[Boolean], msg: c.Expr[Any]) =
    if (assertionsEnabled)
      c.universe.reify(if (!cond.splice) raise(msg.splice))
    else
      c.universe.reify(())
\}
\end{semiverbatim}

\begin{itemize}
\item Seamless integration into existing language
\item Macros use compiler API to create abstract syntax trees
\item \texttt{reify} implements the notion of quasiquoting
\end{itemize}
\end{frame}

\begin{frame}[fragile]
\frametitle{Macros are powerful}

Currently available:
\begin{itemize}
\item Creating new expressions
\end{itemize}
\vskip25pt
We plan to experiment with:
\begin{itemize}
\item Creating new types
\item Adding fields and methods to existing types
\item Steering type inference and implicit search
\end{itemize}
\end{frame}

\begin{frame}[fragile]
\frametitle{Macros are useful}

\begin{itemize}
\item \text{\color{blue}\href{http://slick.typesafe.com/}{Slick}}
\item \text{\color{blue}\href{http://www.paulbutcher.com/2012/06/scalamock-3-0-preview-release/}{ScalaMock v3}}
\item \text{\color{blue}\href{https://github.com/harrah/xsbt/wiki/Definition-Format-Enhancement.md}{SBT v0.13}}
\item \text{\color{blue}\href{http://mandubian.com/2012/11/11/JSON-inception/}{Play v2.1}}
\item \text{\color{blue}\href{https://github.com/pniederw/expecty}{Expecty}}
\item \text{\color{blue}\href{https://github.com/ochafik/Scalaxy}{Scalaxy}}
\item \text{\color{blue}\href{https://github.com/jonifreeman/sqltyped}{Sqltyped}}
\item \text{\color{blue}\href{https://github.com/paulp/declosurify}{Declosurify}}
\item More ideas at \text{\color{blue}\href{http://scalamacros.org/news/2012/11/05/status-update.html}{scalamacros.org}}
\end{itemize}
\end{frame}

\begin{frame}[fragile]
\frametitle{Macros are viable}

\begin{itemize}
\item Implementation footprint is less than 1kloc
\item And we have already simplified the compiler itself using macros
\item Scala reflection, which exposes compiler internals for macro writers, works good enough to be released (although in experimental status)
\item The SIP committee overseeing additions and changes to Scala is convinced that macros are worth trying out
\end{itemize}
\end{frame}

\begin{frame}[fragile]
\frametitle{Macros are magic}

\begin{itemize}
\item Tree construction is hard, because \texttt{reify} has limited usability: \text{\color{blue}\href{http://stackoverflow.com/questions/13795490/how-to-use-type-calculated-in-scala-macro-in-a-reify-clause}{excellent explanation}} by Travis Brown
\item Symbol manipulation is even harder: resetAttrs cargo cult, check out an \text{\color{blue}\href{http://stackoverflow.com/questions/11208790/how-can-i-reuse-definition-ast-subtrees-in-a-macro}{answer at Stack Overflow}} for details
\item Reflection API is at times lacking: \text{\color{blue}\href{https://github.com/paulp/declosurify/blob/ce6c55b040/src/main/scala/reflection.scala\#L98}{befriend \texttt{asInstanceOf}}}
\item Error messages and debugging for generated code are tricky
\end{itemize}
We acknowledge these problems and will do our best to address them. Our latest developments will be covered in a few minutes. Our long-term plans are outlined in the last section of the talk.
\end{frame}

\begin{frame}[fragile]
\frametitle{}

\vskip40pt
\begin{center}
\text{\color{blue}{\Large{Macro paradise}}}
\end{center}
\end{frame}

\begin{frame}[fragile]
\frametitle{Good news}

Since our last report in November, we have made progress
\end{frame}

\begin{frame}[fragile]
\frametitle{Bad news}

\begin{itemize}
\item Scala 2.10.0 is feature frozen for, oh my, \text{\color{blue}\href{https://issues.scala-lang.org/browse/SI/fixforversion/11309}{already four months}}
\item Scala 2.10.x isn't going to welcome new shiny features due to \text{\color{blue}\href{http://groups.google.com/group/scala-internals/browse_thread/thread/933b37c3228c9a29}{compatibility restrictions}}
\item Scala 2.11.0 is scheduled to happen \text{\color{blue}\href{https://issues.scala-lang.org/browse/SI/fixforversion/\#selectedTab=com.atlassian.jira.plugin.system.project\%3Aversions-panel}{only in a year}}
\end{itemize}
\end{frame}

\begin{frame}[fragile]
\frametitle{Macro paradise}

Will be released together with 2.10.0 (on the first week of January 2013), so far lives in \texttt{scalamacros/kepler}:

\begin{itemize}
\item \texttt{paradise/macros} branch at \texttt{scala/scala}
\item Nightlies easily available in SBT and Maven
\item Code is experimental, but successful features are going be merged into trunk around major releases
\item Just like the good old times of 2.10.0-Mx: we hack macros and reflection, you can use new features and fixes immediately
\end{itemize}
\end{frame}

\begin{frame}[fragile]
\frametitle{}

\vskip40pt
\begin{center}
\text{\color{blue}{\Large{Type macros}}}
\end{center}
\end{frame}

\begin{frame}[fragile]
\frametitle{Type macros}
\begin{semiverbatim}
type H2Db(url: String) = macro impl

object Db extends H2Db("coffees")

val brazilian = Db.Coffees.insert("Brazilian", 99, 0)
Db.Coffees.update(brazilian.copy(price = 10))
println(Db.Coffees.all)
\end{semiverbatim}

\vskip15pt

\begin{itemize}
\item Seamless integration into existing language
\item Wherever you can write types, you can use type macros
\item However you can define with def macros (value parameters, type parameters, overloading, overriding, etc), you can define type macros
\end{itemize}
\end{frame}

\begin{frame}[fragile]
\frametitle{Macro implementation}
\begin{semiverbatim}
type H2Db(url: String) = macro impl

def impl(c: Context)(url: c.Expr[String]) = \{
  val name = c.freshName(c.enclosingImpl.name).toTypeName
  val clazz = ClassDef(..., Template(..., generateCode()))
  c.introduceTopLevel(c.enclosingPackage.pid.toString, clazz)
  val classRef = Select(c.enclosingPackage.pid, name)
  Apply(classRef, List(Literal(Constant(c.eval(url)))))
\}

object Db extends H2Db("coffees")
\end{semiverbatim}

\begin{itemize}
\item An entire class gets generated and inserted into the symbol table
\item Macro itself expands into a constructor call, as if the user has written \texttt{object Db extends Db\$1("coffees")}
\item Full source code \text{\color{blue}\href{https://github.com/xeno-by/typemacros-h2db}{at Github}}
\end{itemize}
\end{frame}

\begin{frame}[fragile]
\frametitle{Features}

\begin{itemize}
\item Can be used \text{\color{blue}\href{https://github.com/scalamacros/kepler/blob/paradise/macros/test/files/run/macro-typemacros-used-in-funny-places-a/Test_2.scala}{wherever a type is expected}}
\item \texttt{c.introduceTopLevel} to generate top-level, i.e. non-nested classes and objects (hint: also available in def macros)
\item When used in parent role, have full control over parents, self-types and members of child classes and objects (there'll be an example shortly)
\end{itemize}
\end{frame}

\begin{frame}[fragile]
\frametitle{Roadmap}

Planned:
\begin{itemize}
\item Erasable types
\item Caching for invocations and generated types
\end{itemize}
\vskip25pt
Out of scope (this will be explored later in annotation macros):
\begin{itemize}
\item Addition of inner classes or objects
\item Manipulation of existing classes or objects except for type macros parent roles
\end{itemize}
\end{frame}

\begin{frame}[fragile]
\frametitle{}

\vskip40pt
\begin{center}
\text{\color{blue}{\Large{Quasiquotes}}}
\end{center}
\end{frame}

\begin{frame}[fragile]
\frametitle{History}

\begin{itemize}
\item January 2012: \text{\color{blue}\href{http://xeno-by.livejournal.com/69111.html}{prototype}} based on \texttt{Code.lift}
\item February 2012: \text{\color{blue}\href{https://github.com/scalamacros/kepler/commit/aa0981ff97c48b4db7c33fa04f1ad35787342b64}{prototype}} based on runtime parsing with \texttt{scala.tools.nsc.Global}, implemented by Natallie Baikevich
\item February 2012: \texttt{reify}, which quickly became the official quasiquoting facility for Scala macros
\item December 2012: \text{\color{blue}\href{https://github.com/scalamacros/kepler/tree/paradise/macros}{milestone}} based on compile-time parsing and reification, implemented by Denys Shabalin
\end{itemize}
\end{frame}

\begin{frame}[fragile]
\frametitle{A motivating example}

\begin{semiverbatim}
class D extends Lifter \{
  def x = 2
  // def asyncX = future \{ 2 \}
\}

val d = new D
d.asyncX onComplete \{
  case Success(x) => println(x)
  case Failure(_) => println("failed")
\}
\end{semiverbatim}

\begin{itemize}
\item \texttt{Lifter} is a type macro
\item It takes the body of the host class (\texttt{Template} in scalac parlance) and for each method adds its async version
\item Full source code \text{\color{blue}\href{https://github.com/xeno-by/typemacros-lifter}{at Github}}
\end{itemize}
\end{frame}

\begin{frame}[fragile]
\frametitle{Macro implementation is "simple"}

\begin{semiverbatim}
case ClassDef(_, _, _, Template(_, _, ctor :: defs)) =>
  val defs1 = defs collect \{
    case DefDef(mods, name, tparams, vparamss, tpt, body) =>
      val tpt1 = if (tpt.isEmpty) tpt else AppliedTypeTree(
        Ident(newTermName("Future")), List(tpt))
      val body1 = Apply(
        Ident(newTermName("future")), List(body))
      val name1 = newTermName("async" + name.capitalize)
      DefDef(mods, name1, tparams, vparamss, tpt1, body1)
  \}
  Template(Nil, emptyValDef, ctor +: defs ::: defs1)
\end{semiverbatim}

\begin{itemize}
\item When \texttt{reify} fails, one has to assemble trees manually
\item The code here is a bit simplified, but it still shows how cumbersome and verbose manual tree construction is.
\end{itemize}
\end{frame}

\begin{frame}[fragile]
\frametitle{Quasiquotes}

Before:
\begin{itemize}
\item Apply(Ident(newTermName("future")), List(body))
\item AppliedTypeTree(Ident(newTermName("Future")), List(tpt))
\item case ClassDef(\_, name, \_, Template(\_, \_, \_ :: defs)) $\Rightarrow$ ...
\end{itemize}
\vskip25pt
After:
\begin{itemize}
\item q"future \{ \$body \}"
\item tq"Future[\$tpt]"
\item case q"class \$name \{ ..\$\{\_ :: defs\} \}" $\Rightarrow$ ...
\end{itemize}
\end{frame}

\begin{frame}[fragile]
\frametitle{Roadmap}

Available:
\begin{itemize}
\item Construction and deconstruction of abstract syntax trees
\item Splicing and matching of lists and lists of lists
\end{itemize}
\vskip25pt
Planned:
\begin{itemize}
\item Error reporting
\item Extensive support for language constructs
\item Hygiene and referential transparency
\end{itemize}
\end{frame}

\begin{frame}[fragile]
\frametitle{}

\vskip40pt
\begin{center}
\text{\color{blue}{\Large{Tentative plans for the future}}}
\end{center}
\end{frame}

\begin{frame}[fragile]
\frametitle{Robust tree manipulation}

\begin{itemize}
\item Hygiene and referential transparency resistant to \texttt{resetAttrs}
\item Fixes for non-idempotencies in typer: \text{\color{blue}\href{https://issues.scala-lang.org/browse/SI-5464}{SI-5464}}
\end{itemize}
\end{frame}

\begin{frame}[fragile]
\frametitle{Better infrastructure}

\begin{itemize}
\item IDE support for debugging expanded code: \text{\color{blue}\href{https://issues.scala-lang.org/browse/SI-5922}{SI-5922}}
\item Sane error messages about malformed expansions: \text{\color{blue}\href{https://issues.scala-lang.org/browse/SI-6822}{SI-6822}}
\item Lifecycle management for macro-produced artifacts: \text{\color{blue}\href{https://issues.scala-lang.org/browse/SI-6752}{SI-6752}}
\end{itemize}

This area is being investigated by Dmitry Naydanov, who's upgraded the macro engine and built a prototype of a macro debugger for Intellij. Scala IDE is also going to eventually support debugging of macro expansions. We're looking forward to incorporating this functionality into macro paradise once it's ready.
\end{frame}

\begin{frame}[fragile]
\frametitle{Implicit macros}
\begin{semiverbatim}
trait Serializer[T] \{
  def write(pickle: Pickle, x: T): Unit
\}

def serialize[T](x: T)(implicit s: Serializer[T]): Pickle

implicit def generator[T]: Serializer[T] = macro impl[T]
def impl[T](c: Context): c.Expr[Serializer[T]] = ...

\end{semiverbatim}

\begin{itemize}
\item Sort of work right now, except for \text{\color{blue}\href{https://issues.scala-lang.org/browse/SI-5923}{SI-5923}}
\item A fix would entail a principled redesign of how macros and type inference interact \text{\color{blue}\href{https://issues.scala-lang.org/browse/SI-6755}{SI-6755}}
\end{itemize}
\end{frame}

\begin{frame}[fragile]
\frametitle{Macro annotations}
\begin{semiverbatim}
class atomic extends MacroAnnotation \{
  def complete(defn: _) = macro("generate a backing field")
  def typeCheck(defn: _) = macro("return defn itself")
\}

@atomic var fld: Int

\end{semiverbatim}

\begin{itemize}
\item Statically-typed analogue of Python's decorators
\item Operates on arbitrary definitions
\item Two-step expansion: macro-level + micro-level
\end{itemize}
\end{frame}

\begin{frame}[fragile]
\frametitle{Untyped macros}
\begin{semiverbatim}
val s = "foo=bar"
s.forAllMatches("""^(?<key>.*?)=(?<value>.*)\$""",
    println("key = \%s, value = \%s".format(key, value)))

def forAllMatches(pattern: String, f: _): Unit = macro impl

\end{semiverbatim}

\begin{itemize}
\item Macro arguments are typechecked before macros are called
\item However sometimes this is inconvenient, especially when one wants to adjust with lexical scope in a macro
\item Type safety isn't subverted, because macro expansions are typechecked as usual
\item \text{\color{blue}\href{https://issues.scala-lang.org/browse/SI-5404}{SI-5405}} tracks progress in this direction
\end{itemize}
\end{frame}

\begin{frame}[fragile]
\frametitle{}

\vskip40pt
\begin{center}
\text{\color{blue}{\Large{Summary}}}
\end{center}
\end{frame}

\begin{frame}[fragile]
\frametitle{Summary}

\begin{itemize}
\item Macro paradise, scheduled to be released during the first week of January 2013, will encapsulate development of new macro features
\item Type macros (beta quality) and quasiquotes (milestone quality) are waiting for the new year to be included in \texttt{paradise/macros} at \texttt{scala/scala}
\item These features will be available shortly after release as SNAPSHOT builds of \texttt{org.scala-lang.macro-paradise}
\item To play with the new functionality before the release, build \text{\color{blue}\href{https://github.com/scalamacros/kepler/tree/paradise/macros}{\texttt{paradise/macros} at \texttt{scalamacros/kepler}}}
\item Future development might include erasable type macros, robust tree manipulation, IDE support, implicit macros, macro annotations, untyped macros
\end{itemize}
\end{frame}
\end{document}