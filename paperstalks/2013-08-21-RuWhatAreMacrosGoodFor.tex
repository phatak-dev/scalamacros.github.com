\documentclass[svgnames,hyperref={bookmarks=false},11pt]{beamer}
\usepackage{polyglossia}
\useoutertheme{infolines}
\setdefaultlanguage{russian}
\newfontfamily\cyrillicfont{Cambria}
\newfontfamily\cyrillicfontsf{Calibri}
\newfontfamily\cyrillicfonttt[Scale=0.8]{Monaco}
\defaultfontfeatures{Ligatures=TeX}
\setmainfont{Calibri}
\setsansfont[Scale=MatchLowercase,Mapping=tex-text]{Calibri}
\setbeamertemplate{headline}{} % removes the headline that infolines inserts
% \setbeamertemplate{footline}{
%   \hfill%
%   \usebeamercolor[fg]{page number in head/foot}%
%   \usebeamerfont{page number in head/foot}%
%   \insertpagenumber\,/\,\insertpresentationendpage\kern1em\vskip2pt%
% }
\setbeamertemplate{footline}{
  \hfill%
  \usebeamercolor[fg]{page number in head/foot}%
  \usebeamerfont{page number in head/foot}%
  \footnotesize\insertpagenumber\kern1em\vskip2pt%
}
\setbeamertemplate{navigation symbols}{}
\setbeamercolor{alerted text}{fg=blue}
\setbeamerfont{alerted text}{series=\bfseries,family=\ttfamily}
\setbeamertemplate{section in toc}{\text{\color{black}{\inserttocsection}}}
\usepackage[parfill]{parskip}
\usepackage[linewidth=0.5pt]{mdframed}
\newmdenv[innerleftmargin=1mm, innerrightmargin=1mm, innertopmargin=-1mm, innerbottommargin=2mm, leftmargin=-1mm, rightmargin=-1mm]{lstlistinglike}
\usepackage{tikz}
\usepackage{graphicx}
\DeclareGraphicsExtensions{.pdf,.png,.jpg}
\usepackage{textcomp}
\usepackage{pifont}
\usetikzlibrary{shapes.arrows}

\tikzset{
    myarrow/.style={
        draw,
        fill=orange,
        single arrow,
        minimum height=3.5ex,
        single arrow head extend=1ex
    }
}
\newcommand{\arrowup}{%
\tikz [baseline=-0.5ex]{\node [myarrow,rotate=90] {};}
}
\newcommand{\arrowdown}{%
\tikz [baseline=-1ex]{\node [myarrow,rotate=-90] {};}
}

\hypersetup{pdfauthor={Евгений Бурмако},pdfsubject={Для чего полезны макросы?},pdftitle={Для чего полезны макросы?}}
\title{Для чего полезны макросы?}

\begin{document}

\title{Для чего полезны макросы?}
\author{Евгений Бурмако}
\institute{\'Ecole Polytechnique F\'ed\'erale de Lausanne \\
           \texttt{http://scalamacros.org/}}
\date{21 августа 2013}
{
\setbeamertemplate{footline}{}
\begin{frame}
  \titlepage
\end{frame}
}

\begin{frame}[fragile]
\frametitle{Для чего полезны макросы?}

\begin{itemize}
\item Генерация кода
\item Статические проверки
\item Предметные языки
\end{itemize}
\end{frame}

\begin{frame}[fragile]
\frametitle{}

\vskip40pt
\begin{center}
\text{\color{blue}{\Large{Введение в макросы}}}
\end{center}
\end{frame}

\begin{frame}[fragile]
\frametitle{Что такое макросы?}

\begin{itemize}
\item Экспериментальная функциональность Скалы 2.10.x и 2.11.0
\item Программист пишет функции, использующие Scala Reflection API
\item Компилятор вызывает эти функции во время компиляции
\end{itemize}
\end{frame}

\begin{frame}[fragile]
\frametitle{Виды макросов}

\begin{itemize}
\item Много способов расширения компилятора \text{\textrightarrow} много видов макросов
\item Макротипы, макро-аннотации, нетипизированные макросы и т.д.
\item В официальной поставке 2.10.x и 2.11.0 есть только макрометоды
\end{itemize}
\end{frame}

\begin{frame}[fragile]
\frametitle{Макрометоды}

\begin{semiverbatim}
log(Error, "does not compute")

                          \arrowdown

if (Config.loggingEnabled)
  Config.logger.log(Error, "does not compute")

\end{semiverbatim}

\begin{itemize}
\item Раскрывают типизированные термы в типизированные термы
\item Результат может содержать произвольные конструкции Скалы
\item Кодогенератор может выполнять произвольные вычисления
\end{itemize}
\end{frame}

\begin{frame}[t, fragile]
\frametitle{Макрометоды}

\begin{semiverbatim}
def log(severity: Severity, msg: String): Unit\only<1>{ = ...







}\only<2->{ = macro impl}

\only<2->{def impl(c: Context)
    (severity: c.Expr[Severity],
     msg: c.Expr[String]): c.Expr[Unit]}\only<2>{ = ...





}\only<3->{ = \{
  import c.universe._
  reify \{
    if (Config.loggingEnabled)
      Config.logger.log(severity.splice, msg.splice)
  \}
\}}

\end{semiverbatim}

\begin{itemize}
\item Сигнатуры макросов выглядят как сигнатуры обычных методов
\only<2->{\item Тела макросов \text{\textendash} ссылки на отдельно определяемые реализации}
\only<3->{\item Реализации используют Scala Reflection API для анализа и синтеза}
\end{itemize}
\end{frame}

\begin{frame}[fragile]
\frametitle{Резюме}

\begin{semiverbatim}
log(Error, "does not compute")

                          \arrowdown

if (Config.loggingEnabled)
  Config.logger.log(Error, "does not compute")

\end{semiverbatim}
\begin{itemize}
\item Макрометоды умеют преобразовывать свои вызовы в новый код
\item Преобразования осуществляются только локально
\item Аргументы должны быть статически типизируемы
\end{itemize}
\end{frame}

\begin{frame}[fragile]
\frametitle{}

\vskip40pt
\begin{center}
\text{\color{blue}{\Large{Генерация кода}}}
\end{center}
\end{frame}

\begin{frame}[fragile]
\frametitle{Генерация кода}

\begin{itemize}
\item Создание нового кода на лету
\item Более удобно и надежно, чем генерировать текст
\item Пока что невозможно создавать глобально видимые классы
\end{itemize}
\end{frame}

\begin{frame}[fragile]
\frametitle{Пример №1: улучшение производительности}

\begin{semiverbatim}
def createArray[T: ClassTag](size: Int, el: T) = \{
  val a = new Array[T](size)
  for (i <- 0 until size) a(i) = el
  a
\}

\end{semiverbatim}

\begin{itemize}
\item На Скале легко писать красивый обобщенный код
\item К сожалению, абстракции зачастую приносят накладные расходы
\item Например, в этом случае трансляция полиморфизма в байткод JVM приведет к боксингу, что сильно ухудшит производительность
\end{itemize}
\end{frame}

\begin{frame}[fragile]
\frametitle{Пример №1: улучшение производительности}

\begin{semiverbatim}
def createArray[@specialized T: ClassTag](...) = \{
  val a = new Array[T](size)
  for (i <- 0 until size) a(i) = el
  a
\}

\end{semiverbatim}

\begin{itemize}
\item Методы можно специализировать, но это нудно и тяжеловесно
\item Нудно = приходится специализировать всю цепочку вызовов
\item Тяжеловесно = специализация приводит к дубликации байткода
\end{itemize}
\end{frame}

\begin{frame}[fragile]
\frametitle{Пример №1: улучшение производительности}

\begin{semiverbatim}
def createArray[T: ClassTag](size: Int, el: T) = \{
  val a = new Array[T](size)
  \alert{def specBody[@specialized T](el: T) \{}
    \alert{for (i <- 0 until size) a(i) = el}
  \alert{\}}
  \alert{classTag[T] match \{}
    \alert{case ClassTag.Int => specBody(el.asInstanceOf[Int])}
    \alert{...}
  \alert{\}}
  a
\}
\end{semiverbatim}

\begin{itemize}
\item Хотелось бы специализировать ровно столько, сколько нужно
\item Прямо как в недавней работе \text{\color{blue}\href{http://lampwww.epfl.ch/\~hmiller/scala2013/resources/pdfs/paper10.pdf}{Bridging Islands of Specialized Code}}
\item Также хотелось бы не писать подобный низкоуровневый код руками
\item И вот здесь оказываются очень полезными макросы!
\end{itemize}
\end{frame}

\begin{frame}[fragile]
\frametitle{Пример №1: улучшение производительности}

\begin{semiverbatim}
\alert{def specialized[T: ClassTag](code: => T) = macro ...}

def createArray[T: ClassTag](size: Int, el: T) = \{
  val a = new Array[T](size)
  \alert{specialized[T] \{}
    for (i <- 0 until size) a(i) = el
  \alert{\}}
  a
\}
\end{semiverbatim}

\begin{itemize}
\item Макрос \texttt{specialized} берет красивый код и делает его быстрым
\item Это типичный сценарий применения ускоряющих макросов
\item \text{\color{blue}\href{https://issues.scala-lang.org/browse/SI-6711}{Иногда такие макросы нетривиальны}}, но прогресс не стоит на месте
\end{itemize}
\end{frame}

\begin{frame}[fragile]
\frametitle{Пример №2: материализация}

\begin{semiverbatim}
trait Reads[T] \{
  def reads(json: JsValue): JsResult[T]
\}

object Json \{
  def fromJson[T](json: JsValue)
    (implicit fjs: Reads[T]): JsResult[T]
\}

\end{semiverbatim}

\begin{itemize}
\item Классы типов \text{\textendash} красивый подход к написанию расширяемого кода
\item Вот пример сериализатора из Play, основанного на классах типов
\end{itemize}
\end{frame}

\begin{frame}[fragile]
\frametitle{Пример №2: материализация}

\begin{semiverbatim}
def fromJson[T](json: JsValue)
  (implicit fjs: Reads[T]): JsResult[T]

implicit val IntReads = new Reads[Int] \{
  def reads(json: JsValue): JsResult[T] = ...
\}

fromJson[Int](json) // пишем
fromJson[Int](json)(IntReads) // получаем

\end{semiverbatim}

\begin{itemize}
\item Классы типов абстрагируют варьирующиеся части алгоритмов
\item Для каждого используемого типа определяются экземпляры
\item Компилятор автоматически подставляет их в нужные вызовы
\end{itemize}
\end{frame}

\begin{frame}[fragile]
\frametitle{Пример №2: без макросов (Скала 2.9.х)}

\begin{semiverbatim}
case class Person(name: String, age: Int)

implicit val personReads = (
  (__ \\ 'name).reads[String] and
  (__ \\ 'age).reads[Int]
)(Person)

\end{semiverbatim}

\begin{itemize}
\item Экземпляры классов типов пишутся вручную
\item Много некрасивого, избыточного кода
\item Есть \text{\color{blue}\href{https://speakerdeck.com/larsrh/generating-type-class-instances-automatically}{альтернативы}}, но у них есть свои недостатки
\end{itemize}
\end{frame}

\begin{frame}[fragile]
\frametitle{Пример №2: макрометоды (Скала 2.10.0)}

\begin{semiverbatim}
implicit val personReads = Json\alert{.reads[}Person\alert{]}

\end{semiverbatim}

\begin{itemize}
\item Как мы уже видели, избыточный код можно генерировать
\item Байткод будет идентичен байткоду предыдущего примера
\item Поэтому производительность останется на отличном уровне
\end{itemize}
\end{frame}

\begin{frame}[fragile]
\frametitle{Пример №2: неявные макросы (Скала 2.10.2+)}

\begin{semiverbatim}
// не требуется вообще никакого кода

\end{semiverbatim}

\begin{itemize}
\item Экземпляры классов типов можно генерировать на лету
\item Поэтому объявлять неявные значения нет необходимости
\item Подход применяется в таких библиотеках как \text{\color{blue}\href{http://github.com/scala/pickling}{Pickling}}
и \text{\color{blue}\href{http://github.com/milessabin/shapeless}{Shapeless}}
\end{itemize}
\end{frame}

\begin{frame}[fragile]
\frametitle{Пример №2: неявные макросы (Скала 2.10.2+)}

\begin{semiverbatim}
trait Reads[T] \{ def reads(json: JsValue): JsResult[T] \}

object Reads \{
  \alert{implicit def materializeReads[T]: Reads[T] = macro ...}
\}

\end{semiverbatim}

\begin{itemize}
\item Поиск неявных значений затрагивает не только область видимости
\item Но также и список полей и методов компаньонов
\item Поэтому неявный макрос, объявленный в компаньоне, будет виден всем, кто захочет использовать наш класс типов.
\end{itemize}
\end{frame}

\begin{frame}[fragile]
\frametitle{Пример №2: неявные макросы (Скала 2.10.2+)}

\begin{semiverbatim}
fromJson[Person](json)

                          \arrowdown

fromJson[Person](json)(\alert{materializeReads[}Person\alert{]})

                          \arrowdown

fromJson[Person](json)(new Reads[Person]\{ ... \})

\end{semiverbatim}

\begin{itemize}
\item Если экземпляр \texttt{Reads[T]} не находится, будет вызываться макрос
\item Макрос посмотрит на значение \texttt{T} и сгенерирует нужный экземпляр
\item Подробности можно почитать в \text{\color{blue}\href{http://www.meetup.com/Bay-Area-Scala-Enthusiasts/events/121848382/}{слайдах другого выступления}}
\end{itemize}
\end{frame}

\begin{frame}[fragile]
\frametitle{Пример №3: поставщики типов}

\begin{semiverbatim}
println(Db.Coffees.all)
Db.Coffees.insert("Brazilian", 99, 0)

\end{semiverbatim}

\begin{itemize}
\item F\# позволяет генерировать обертки вокруг источников данных
\item Будучи статически типизированными, эти обертки приятны и удобны
\item Можно ли что-то похожее реализовать на макрометодах?
\end{itemize}
\end{frame}

\begin{frame}[fragile, t]
\frametitle{Пример №3: поставщики типов}

\begin{semiverbatim}
\alert{def h2db(connString: String): Any = macro ...}
val db = \alert{h2db(}"jdbc:h2:coffees.h2.db"\alert{)}

                          \arrowdown

\end{semiverbatim}
\end{frame}

\begin{frame}[fragile, t]
\frametitle{Пример №3: поставщики типов}

\begin{semiverbatim}
\alert{def h2db(connString: String): Any = macro ...}
val db = \alert{h2db(}"jdbc:h2:coffees.h2.db"\alert{)}

                          \arrowdown

val db = \{
  trait Db \{
    case class Coffee(...)
    val Coffees: Table[Coffee] = ...
  \}
  new Db \{\}
\}
\end{semiverbatim}

\begin{itemize}
\item Макрометоды раскрываются локально
\item Поэтому максимум, чего можно добиться в данном случае, это набор локальных классов, которые не видны извне
\end{itemize}
\end{frame}

\begin{frame}[fragile, t]
\frametitle{Пример №3: поставщики типов}

\begin{semiverbatim}
scala> val db = h2db("jdbc:h2:coffees.h2.db")
db: AnyRef \{
  type Coffee \{ val name: String; val price: Int; ... \}
  val Coffees: Table[this.Coffee]
\} = \$anon\$1...

scala> db.Coffees.all
res1: List[Db\$1.this.Coffee] = List(Coffee(Brazilian,99,0))

\end{semiverbatim}

\begin{itemize}
\item К счастью, локальные классы стираются в структурные типы
\item Поэтому все работает как надо (статическая типизация, IDE)
\item Единственная проблема - накладные расходы на рефлексию
\end{itemize}
\end{frame}

\begin{frame}[fragile, t]
\frametitle{Пример №3: поставщики типов}

\begin{semiverbatim}
class Coffee(row: Row[\alert{"...".type}]) with Dynamic \{
  \alert{def selectDynamic = macro ...}
\}

db: AnyRef\{type Coffee <: Dynamic; ...\}
coffee\alert{.}name // преобразуется в: coffee.\alert{selectDynamic(}"name"\alert{)}

                          \arrowdown

coffee.row["name"].asInstanceOf[String]

\end{semiverbatim}

\begin{itemize}
\item Оказывается, \text{\color{blue}\href{http://meta.plasm.us/posts/2013/07/11/fake-type-providers-part-2/}{можно обойтись без структурных типов}}
\item Динамическая типизация + макросы = статическая типизация
\item Этот подход \text{\color{blue}\href{http://meta.plasm.us/posts/2013/07/12/vampire-methods-for-structural-types/}{можно улучшать и дальше}}
\end{itemize}
\end{frame}

\begin{frame}[fragile]
\frametitle{Пример №3: честные поставщики типов}

\begin{semiverbatim}
\alert{@H2Db(}"jdbc:h2:coffees.h2.db"\alert{)} object Db
println(Db.Coffees.all)
Db.Coffees.insert("Brazilian", 99, 0)

\end{semiverbatim}

\begin{itemize}
\item Эмуляция поставщиков типов на структурных типах это прикольно
\item Но ее стоит воспринимать только как временное решение
\item \text{\color{blue}\href{http://docs.scala-lang.org/overviews/macros/annotations.html}{Макро аннотации}} должны закрыть этот вопрос раз и навсегда
\end{itemize}
\end{frame}

\begin{frame}[fragile]
\frametitle{}

\vskip40pt
\begin{center}
\text{\color{blue}{\Large{Статические проверки}}}
\end{center}
\end{frame}

\begin{frame}[fragile]
\frametitle{Статические проверки}

\begin{itemize}
\item Поиск ошибок в программе во время компиляции
\item Можно выдавать собственные ошибки и предупреждения
\item Невозможно выполнять глобальные проверки
\end{itemize}
\end{frame}

\begin{frame}[fragile]
\frametitle{Пример №4: статически типизированные строки}

\begin{semiverbatim}
scala> val x = "42"
x: String = 42

scala> "%d".format(x)
j.u.IllegalFormatConversionException: d != java.lang.String
  at java.util.Formatter\$FormatSpecifier.failConversion...
\only<2>{
scala> f"\$x\%d"
<console>:31: error: type mismatch;
 found   : String
 required: Int
}
\end{semiverbatim}

\begin{itemize}
\item Строки обычно считаются небезопасными
\only<2>{\item Но, когда за дело берутся макросы, все меняется}
\only<2>{\item Особенно если использовать строковую интерполяцию}
\end{itemize}
\end{frame}

\begin{frame}[fragile]
\frametitle{Пример №4: статически типизированные строки}

\begin{semiverbatim}
implicit class Formatter(c: StringContext) \{
  \alert{def f(args: Any*): String = macro ...}
\}

val x = "42"
f"\$x\%d" // преобразуется в: StringContext("", "%d").\alert{f(}x\alert{)}

                          \arrowdown

val arg\$1: Int = x \alert{// ошибка компиляции}
"%d".format(arg\$1)

\end{semiverbatim}

\begin{itemize}
\item Макрос \texttt{f} превращает отформатированные строки в блоки кода
\item Явные аннотации типов гарантируют отсутствие ошибок
\item Похожие техники есть для регулярных выражений, двоичных литералов и т.д.
\end{itemize}
\end{frame}

\begin{frame}[fragile, t]
\frametitle{Пример №5: типизированные каналы Акки}

\begin{semiverbatim}
trait Request
case class Command(msg: String) extends Request

trait Reply
case object CommandSuccess extends Reply
case class CommandFailure(msg: String) extends Reply

\only<1>{val actor = someActor}\only<2>{type Spec = (Request, Reply) :+: TNil}
\only<1>{actor ! Command}\only<2>{val actor = new ChannelRef[Spec](someActor)}
\alert{\only<2>{actor <-!- Command // ошибка компиляции}}

\end{semiverbatim}

\begin{itemize}
\only<1>{\item Актеры в Акке динамически типизированы, т.е. \texttt{!} принимает \texttt{Any}}
\only<1>{\item Это делает невозможным статическую проверку сообщений}
\only<2>{\item В принципе, сигнатуры актеров можно описать и без макросов}
\only<2>{\item Но практичным это становится только на макросах}
\only<2>{\item \text{\color{blue}\href{http://doc.akka.io/docs/akka/snapshot/scala/typed-channels.html}{Типизированные каналы Акки}} разработаны специально для этого}
\end{itemize}
\end{frame}

\begin{frame}[fragile]
\frametitle{Пример №5: типизированные каналы Акки}

\begin{semiverbatim}
type Spec = (Request, Reply) :+: TNil
val actor = new ChannelRef[Spec](someActor)
\alert{actor <-!- Command // ошибка компиляции}

\end{semiverbatim}

\begin{itemize}
\item Макрос \texttt{<-!-} берет тип префикса и извлекает спецификацию канала
\item Потом он вычисляет тип аргумента и проверяет его на соответствие
\item Все это можно реализовать при помощи вычислений на типах
\item Но это будет очень сложно и для программиста, и для пользователей
\end{itemize}
\end{frame}

\begin{frame}[fragile, t]
\frametitle{Пример №6: споры}

\begin{semiverbatim}
def future[T](body: => T) = ...

def receive = \{
  case Request(data) =>
    future \{
      val result = transform(data)
      sender ! Response(result)
    \}
\}

\end{semiverbatim}

\begin{itemize}
\item Пример выше иллюстрирует распространенную ошибку
\item Захват переменной \texttt{sender} в замыкание \text{\textendash} это плохая идея
\item \texttt{sender} это не значение, как может показаться, а изменяемый метод
\item Такого рода ошибки можно ловить макросами: \text{\color{blue}\href{http://docs.scala-lang.org/sips/pending/spores.html}{SIP-21 \text{\textendash} Spores}}
\end{itemize}
\end{frame}

\begin{frame}[fragile, t]
\frametitle{Пример №6: споры}

\begin{semiverbatim}
def future[T](body: \alert{Spore[T]}) = ...

\alert{def spore[T](body: => T): Spore[T] = macro ...}

def receive = \{
  case Request(data) =>
    future(\alert{spore \{}
      val result = transform(data)
      sender ! Response(result) \alert{// ошибка компиляции}
    \alert{\}})
\}

\end{semiverbatim}

\begin{itemize}
\item Макрос \texttt{spore} вычисляет свободные переменные в своем аргументе
\item Если какая-либо из этих переменных изменяемая, выдается ошибка
\end{itemize}
\end{frame}

\begin{frame}[fragile, t]
\frametitle{Пример №6: споры}

\begin{semiverbatim}
def future[T](body: Spore[T]) = ...

\alert{implicit def anyToSpore[T](body: => T): Spore[T] = macro ...}

def receive = \{
  case Request(data) =>
    future \alert{\{}
      val result = transform(data)
      sender ! Response(result) \alert{// ошибка компиляции}
    \alert{\}}
\}

\end{semiverbatim}

\begin{itemize}
\item Преобразование в споры можно сделать неявным
\item В этом случае проверка замыканий будет совершенно незаметной
\end{itemize}
\end{frame}

\begin{frame}[fragile]
\frametitle{}

\vskip40pt
\begin{center}
\text{\color{blue}{\Large{Предметные языки}}}
\end{center}
\end{frame}

\begin{frame}[fragile]
\frametitle{Мечта разработчиков предметных языков}

\begin{itemize}
\item Пользователь пишет обычную программу
\item Компилятор прозрачно превращает ее в структуру данных
\end{itemize}
\end{frame}

\begin{frame}[fragile]
\frametitle{Мечта разработчиков предметных языков}

\begin{itemize}
\item Даже из коробки Скала очень хороша
\item \text{\color{blue}\href{http://scala-lms.github.io/}{LMS}} (также известный как Scala Virtualized) делает Скалу еще лучше
\item Однако LMS тяжеловесен и не работает с официальной Скалой
\item Макросы предоставляют легковесную альтернативу
\end{itemize}
\end{frame}

\begin{frame}[fragile, t]
\frametitle{Пример №7: LINQ}

\begin{semiverbatim}
val usersMatching = query[String, (Int, String)](
  "select id, name from users where name = ?")
usersMatching("John")

\only<2->{case class User(id: Column[Int], name: Column[String])}
\only<2->{users.filter(_.name === "John")}

\only<3->{case class User(id: Int, name: String)}
\only<3->{users\alert{.filter(}_.name == "John"\alert{)}}

\end{semiverbatim}

\begin{itemize}
\item Запросы к базам данных можно писать на SQL
\only<2->{\item Их также можно закодировать в DSL, правда с некоторым трудом}
\only<3->{\item Или же, благодаря макросам, можно все писать на обычной Скале}
\end{itemize}
\end{frame}

\begin{frame}[fragile]
\frametitle{Пример №7: LINQ}

\begin{semiverbatim}
trait Query[T] \{
  \alert{def filter(p: T => Boolean): Query[T] = macro ...}
\}

val users: Query[User] = ...
users\alert{.filter(}_.name == "John"\alert{)}


                          \arrowdown

Query(Filter(users, Equals(Ref("name"), Literal("John"))))

\end{semiverbatim}

\begin{itemize}
\item Макрос \texttt{filter} превращает вызовы методов в структуры данных
\item Дальше эти структуры можно транслировать в SQL
\item Можно делать и по-другому: \text{\color{blue}\href{https://github.com/amirsh/master-thesis}{An Embedded Query Language in Scala}}
\end{itemize}
\end{frame}

\begin{frame}[fragile, t]
\frametitle{Пример №8: асинхронные вычисления}

\begin{semiverbatim}
val futureDOY: Future[Response] =
  WS.url("http://api.day-of-year/today").get

val futureDaysLeft: Future[Response] =
  WS.url("http://api.days-left/today").get

futureDOY.flatMap \{ doyResponse =>
  val dayOfYear = doyResponse.body
  futureDaysLeft.map \{ daysLeftResponse =>
    val daysLeft = daysLeftResponse.body
    Ok(s"\$dayOfYear: \$daysLeft days left!")
  \}
\}
\end{semiverbatim}

\begin{itemize}
\item Сделать программу асинхронной совсем непросто
\item По сути, приходится выворачивать поток управления наизнанку
\end{itemize}
\end{frame}

\begin{frame}[fragile, t]
\frametitle{Пример №8: асинхронные вычисления}

\begin{semiverbatim}
\alert{def async[T](body: => T): Future[T] = macro ...}
\alert{def await[T](future: Future[T]): T = macro ...}

\alert{async \{}
  val dayOfYear = \alert{await(}futureDOY\alert{)}.body
  val daysLeft = \alert{await(}futureDaysLeft\alert{)}.body
  Ok(s"\$dayOfYear: \$daysLeft days left!")
\alert{\}}



\end{semiverbatim}

\begin{itemize}
\item Сделать программу асинхронной совсем непросто
\item Но макросы могут сделать это автоматически: \text{\color{blue}\href{http://docs.scala-lang.org/sips/pending/async.html}{SIP-22 \text{\textendash} Async}}
\item Генераторы из C\# и Python эмулируются похожим образом
\end{itemize}
\end{frame}

\begin{frame}[fragile, t]
\frametitle{Пример №9: Datomisca}
\begin{semiverbatim}
scala> \alert{Query(}"""
|      [ :find ?e ?n
|        :in \$ ?char
|        :where  [ ?e :person/name ?n ]
|                [ ?e person/character ?char ]
|      ]
|      """\alert{)}
res0: TypedQueryAuto2[DatomicData, DatomicData, (DatomicData,
DatomicData)] = [ :find ?e ?n :in \$ ?char :where ... ]

\end{semiverbatim}

\begin{itemize}
\item Благодаря макросам строковая интерполяция становится механизмом встраивания внешних языков
\item В этом примере в Скалу были интегрированы \text{\color{blue}\href{http://pellucidanalytics.github.io/datomisca/index.html}{запросы к Datomic}}
\item Почти даром получаются статические проверки синтаксиса и типов
\item Механизм можно обобщить до \text{\color{blue}\href{https://github.com/densh/talks/raw/master/ecoop-2013-07-02/index.pdf}{Modular Quasiquote Abstraction}}
\end{itemize}
\end{frame}

\begin{frame}[fragile]
\frametitle{}

\vskip40pt
\begin{center}
\text{\color{blue}{\Large{Заключение}}}
\end{center}
\end{frame}

\begin{frame}[fragile]
\frametitle{Для чего полезны макросы?}

\begin{itemize}
\item Генерация кода
\item Статические проверки
\item Предметные языки
\end{itemize}
\end{frame}

\end{document}
