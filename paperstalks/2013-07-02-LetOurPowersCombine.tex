\documentclass[svgnames,hyperref={bookmarks=false}]{beamer}
\usepackage[T1]{fontenc}
\useoutertheme{infolines}
\setbeamertemplate{headline}{} % removes the headline that infolines inserts
% \setbeamertemplate{footline}{
%   \hfill%
%   \usebeamercolor[fg]{page number in head/foot}%
%   \usebeamerfont{page number in head/foot}%
%   \insertpagenumber\,/\,\insertpresentationendpage\kern1em\vskip2pt%
% }
\setbeamertemplate{footline}{
  \hfill%
  \usebeamercolor[fg]{page number in head/foot}%
  \usebeamerfont{page number in head/foot}%
  \footnotesize\insertpagenumber\kern1em\vskip2pt%
}
\setbeamertemplate{navigation symbols}{}
\setbeamercolor{alerted text}{fg=blue}
\setbeamerfont{alerted text}{series=\bfseries,family=\ttfamily}
\setbeamertemplate{section in toc}{\text{\color{black}{\inserttocsection}}}
\usepackage[parfill]{parskip}
\usepackage[linewidth=0.5pt]{mdframed}
\newmdenv[innerleftmargin=1mm, innerrightmargin=1mm, innertopmargin=-1mm, innerbottommargin=2mm, leftmargin=-1mm, rightmargin=-1mm]{lstlistinglike}
\usepackage{tikz}
\usepackage{graphicx}
\DeclareGraphicsExtensions{.pdf,.png,.jpg}
\usepackage{textcomp}
\usepackage{pifont}
\usetikzlibrary{shapes.arrows}

\tikzset{
    myarrow/.style={
        draw,
        fill=orange,
        single arrow,
        minimum height=3.5ex,
        single arrow head extend=1ex
    }
}
\newcommand{\arrowup}{%
\tikz [baseline=-0.5ex]{\node [myarrow,rotate=90] {};}
}
\newcommand{\arrowdown}{%
\tikz [baseline=-1ex]{\node [myarrow,rotate=-90] {};}
}
\tikzset{
    mydoublearrow/.style={
        draw,
        fill=orange,
        double arrow,
        minimum height=5.5ex,
        double arrow head extend=1ex
    }
}
\newcommand{\arrowupdown}{%
\tikz [baseline=-0.5ex]{\node [mydoublearrow,rotate=90] {};}
}

\hypersetup{pdfauthor={Eugene Burmako},pdfsubject={Let Our Powers Combine!},pdftitle={Let Our Powers Combine!}}
\title{Let Our Powers Combine!}

\begin{document}

\author{Eugene Burmako}
\institute{\'Ecole Polytechnique F\'ed\'erale de Lausanne \\
           \texttt{http://scalamacros.org/}}
\date{02 July 2013}
{
\setbeamertemplate{footline}{}
\begin{frame}
  \titlepage
\end{frame}
}

\begin{frame}[fragile]
\frametitle{Agenda}

\begin{itemize}
\item Compile-time metaprogramming with macros
\item Integration with rich syntax and static types
\item Hammers: a number of macro flavors for Scala
\item Nails: case studies for the macro flavors
\item No details of the underlying macro system
\end{itemize}
\end{frame}

\begin{frame}[fragile]
\frametitle{}

\vskip40pt
\begin{center}
\text{\color{blue}{\Large{Hammers: The Macro Flavors}}}
\end{center}
\end{frame}

\begin{frame}[fragile]
\frametitle{Macros}

\begin{itemize}
\item Realize textual abstraction in Scala 2.10.0+
\item Written in Scala against the Scala reflection API
\item Invoked by the compiler during compilation
\item Influence compilation: change code, affect types, etc
\end{itemize}
\end{frame}

\begin{frame}[fragile]
\frametitle{Macro flavors}

\begin{itemize}
\item Scala has rich syntax
\item It distinguishes terms and types, expressions and definitions
\item To abstract over syntax we take that into account:
\begin{itemize}
\item Terms => def macros
\item Types => type macros
\item Definitions => macro annotations
\end{itemize}
\end{itemize}
\end{frame}

\begin{frame}[fragile]
\frametitle{Def macros}

\begin{semiverbatim}
printf("hello \%s", "world!")

                          \arrowdown

def tmp\$1: String = "world!"
print("hello ")
print(tmp\$1)

\end{semiverbatim}

\begin{itemize}
\item Def macros syntactically replace terms with other terms
\item Generated code might contain arbitrary Scala constructs
\item Code generation might involve arbitrary computations
\end{itemize}
\end{frame}

\begin{frame}[fragile]
\frametitle{Def macros}

\begin{semiverbatim}
def printf(format: String, args: Any*): Unit = macro impl

def impl(c: Context)(
      format: c.Expr[String],
      args: c.Expr[Any]*): c.Expr[Unit] = \{
  \only<2->{import c.universe._}
  \only<2->{val q"\$\{sFormat: String\}" = format}
  \only<2->{val (defs, parts) = parse(sFormat, args)}
  \only<3->{val stmts = defs ++ parts.map(part => q"print(\$part)")}
  \only<3->{q"..\$stmts"}
\}

\end{semiverbatim}

\begin{itemize}
\item Def macros = definitions + implementations
\item Definitions look and feel like normal Scala methods (almost!)
\item Implementations are code-transforming metaprograms
\end{itemize}
\end{frame}

\begin{frame}[fragile]
\frametitle{Type macros}

\begin{semiverbatim}
object Db extends H2Db("jdbc:h2:coffees.h2.db")

                          \arrowdown

@synthetic trait CoffeesH2Db\$1 \{
  case class Coffee(...)
  val Coffees: Table[Coffee] = ...
  ...
\}
object Db extends CoffeesH2Db\$1

\end{semiverbatim}

\begin{itemize}
\item Type macros syntactically replace types with other types
\item Can generate auxiliary classes or objects in the process
\end{itemize}
\end{frame}

\begin{frame}[fragile]
\frametitle{Type macros}

\begin{semiverbatim}
type H2Db(connString: String) = macro impl

def impl(c: Context)(connString: c.Expr[String]) = \{
  \only<2->{val schema = loadSchema(connString)}
  \only<2->{val name = schema.dbName + "H2Db"}
  \only<2->{val members = generateCode(schema)}
  \only<3->{c.introduce(q"@synthetic trait \$name \{ \$members \}")}
  \only<3->{q"\$name"}
\}

\end{semiverbatim}

\begin{itemize}
\item Type macros = definitions + implementations
\item Definitions look and feel like normal Scala types (almost!)
\item Implementations are code-transforming metaprograms
\end{itemize}
\end{frame}

\begin{frame}[fragile]
\frametitle{Macro annotations}

\begin{semiverbatim}
@serializable class C(x: Int)

                          \arrowdown

class C(x: Int) \{
  def serialize = ...
\}

\end{semiverbatim}

\begin{itemize}
\item Macro annotations syntactically replace definitions with other definitions
\end{itemize}
\end{frame}

\begin{frame}[fragile]
\frametitle{Macro annotations}

\begin{semiverbatim}
class serializable extends MacroAnnotation \{
  def transform = macro impl
\}

def impl(c: Context) = \{
  \only<2->{val logic = generateCode(c.annottee)}
  \only<2->{val serialize = q"def serialize = \$logic"}
  \only<3->{val q"class \$name(\$params) \{ \$members \}" = c.annottee}
  \only<3->{q"class \$name(\$params) \{ \${members :+ serialize} \}"}
\}

\end{semiverbatim}

\begin{itemize}
\item Macro annotations = definitions + implementations
\item Definitions look and feel like normal Scala annotations (almost!)
\item Implementations are code-transforming metaprograms
\end{itemize}
\end{frame}

\begin{frame}[fragile]
\frametitle{Overarching theme}

\begin{itemize}
\item Metaprograms are hidden behind vanilla Scala features
\item This integrates macros into the language in a very natural way
\item Therefore allowing to enrich existing features with new meanings
\end{itemize}
\end{frame}

\begin{frame}[fragile]
\frametitle{Synergy with rich syntax and static types}

\begin{itemize}
\item Scala builds a lot of features on top of others
\begin{itemize}
\item Application: \texttt{apply}
\item Getters and setters: \texttt{foo} and \texttt{foo\_=}
\item For comprehensions: \texttt{flatMap}, \texttt{map}, \texttt{withFilter}, \texttt{foreach}
\item Dynamic: \texttt{selectDynamic}, \texttt{updateDynamic}, \texttt{applyDynamic}
\item String interpolation: extension methods on \texttt{StringContext}
\item Implicits: \texttt{implicit} modifier on methods
\end{itemize}
\item All those can be defined as macros, gaining
\begin{itemize}
\item Compile-time programmability
\item Code generation powers
\end{itemize}
\end{itemize}
\end{frame}

\begin{frame}[fragile]
\frametitle{}

\vskip40pt
\begin{center}
\text{\color{blue}{\Large{Nails: The Macro Applications}}}
\end{center}
\end{frame}

\begin{frame}[fragile]
\frametitle{}

\vskip40pt
\begin{center}
\text{\color{blue}{\Large{Language virtualization}}}
\end{center}
\end{frame}

\begin{frame}[fragile]
\frametitle{Language virtualization}

\begin{semiverbatim}
coffees.filter(c => c.price < 10)

                          \arrowdown

coffees.filter(LessThan(Ref("price"), Literal(10)))





\end{semiverbatim}

\begin{itemize}
\item Take a code snippet written in Scala and represent it as data
\item Then interpret this data, potentially with different semantics
\item Can be used for deep embedding of internal DSLs
\end{itemize}
\end{frame}

\begin{frame}[fragile, t]
\frametitle{Language virtualization}

\begin{semiverbatim}
case class Queryable[T](val query: Query) \{
  def filter(p: T => Boolean): Queryable[T] =
    macro QueryableMacros.filter[T]

  def toList: List[T] = \{
    val translatedQuery = query.translate
    translatedQuery.execute.asInstanceOf[List[T]]
  \}
  ...
\}

\end{semiverbatim}

\begin{itemize}
\item Query operators can be implemented as macros
\item These macros have their arguments lifted automatically
\item Lifted arguments can then be remembered and accumulated
\item No additional language features are necessary
\end{itemize}
\end{frame}

\begin{frame}[fragile, t]
\frametitle{Language virtualization}

\begin{semiverbatim}
object QueryableMacros \{
  def filter[T: c.TypeTag](c: Context)(p: c.Tree) = \{
    \only<1->{import c.universe._}
    \only<1->{val lifted: c.Tree = QueryableMacros.lift(p)}
    \only<2->{val T: c.Type = typeOf[T]}
    \only<2->{val callee: c.Tree = c.prefix}
    \only<2->{q"Queryable[\$T](\$callee.query.filter(\$lifted))"}
  \}
  ...
\}

\end{semiverbatim}

\begin{itemize}
\item Query operators can be implemented as macros
\item These macros have their arguments lifted automatically
\item Lifted arguments can then be remembered and accumulated
\item No additional language features are necessary
\end{itemize}
\end{frame}

\begin{frame}[fragile]
\frametitle{Comparison with staging}

\begin{itemize}
\item Macros allow for earlier error detection
\item Macros don't need stage annotations
\item Staging composes better
\end{itemize}
\end{frame}

\begin{frame}[fragile]
\frametitle{Composability}

\begin{semiverbatim}
case class Coffee(name: String, price: Double)
val coffees: Queryable[Coffee] = Db.coffees

// closed world
coffees.filter(c => c.price < 10)

// open world
def isAffordable(c: Coffee) = c.price < 10
scoffees.filter(isAffordable)

\end{semiverbatim}

\begin{itemize}
\item Code is only lifted within macro arguments
\item Therefore separate compilation doesn't work out of the box
\begin{itemize}
\item Decompilation to recreate abstract syntax trees
\item \texttt{@lifted} macro annotation to retain abstract syntax trees
\end{itemize}
\end{itemize}
\end{frame}

\begin{frame}[fragile]
\frametitle{}

\vskip40pt
\begin{center}
\text{\color{blue}{\Large{Type providers}}}
\end{center}
\end{frame}

\begin{frame}[fragile]
\frametitle{Type providers}

\begin{semiverbatim}
type NetFlix = ODataService<\textquotedbl...\textquotedbl>
let netflix = NetFlix.GetDataContext()
let avatarTitles = query \{
  for t in netflix.Titles do
  where (t.Name.Contains "Avatar") sortBy t.Name take 100
\}

\end{semiverbatim}
\begin{itemize}
\item F\# features type providers, a compile-time facility for code generation
\item Type providers can be mostly reimplemented with macros
\end{itemize}
\end{frame}
\begin{frame}[fragile]

\frametitle{Picking a macro flavor}

\begin{semiverbatim}
object Db extends H2Db("jdbc:h2:coffees.h2.db")

                          \arrowupdown

@H2Db("jdbc:h2:coffees.h2.db")
object Db

\end{semiverbatim}

\begin{itemize}
\item Type macros provide very similar codegen capabilities to type providers
\item But thanks to first-class modules macro annotations also fit the bill
\end{itemize}
\end{frame}

\begin{frame}[fragile, t]
\frametitle{Erased type providers}

\begin{semiverbatim}
object Netflix \{
  case class Title(name: String)
  def Titles: List[Title] = ...

  case class Director(name: String)
  def Directors: List[Director] = ...

  ...
\}

\end{semiverbatim}

\begin{itemize}
\item Sometimes being strongly-typed is too wasteful
\item If data comes in untyped anyway, it might be unnecessary to wrap it
\item In F\# one can avoid unnecessary classes using erased type providers
\item In Scala we can't reimplement them, but we can emulate
\end{itemize}
\end{frame}

\begin{frame}[fragile, t]
\frametitle{Emulating erasure}

\begin{semiverbatim}
object Netflix \{
  type Title = XmlEntity\only<2->{["http://.../Title".type]}
  def Titles: List[Title] = ...

  type Director = XmlEntity\only<2->{["http://.../Director".type]}
  def Directors: List[Director] = ...

  ...
\}

\end{semiverbatim}

\begin{itemize}
\only<1>{\item We want to replace strongly-typed wrappers with underlying classes}
\only<1>{\item And not to lose static typing like in the code of the example}
\only<1>{\item This is possible due to synergies of macros and vanilla Scala features!}
\only<2->{\item Wrappers are replaced with type aliases parameterized by identities}
\only<2->{\item (The \texttt{"...".type} cannot be written, but can be generated!)}
\only<2->{\item All selections and calls are then statically intercepted with macros}
\end{itemize}
\end{frame}

\begin{frame}[fragile]
\frametitle{Emulating erasure}

\begin{semiverbatim}
class XmlEntity[Url] extends Dynamic \{
  def selectDynamic(field: String) = macro XmlEntity.impl
\}

object XmlEntity \{
  def impl(c: Context)(field: c.Tree) = \{
    \only<2->{import c.universe._}
    \only<2->{val TypeRef(_, _, tUrl) = c.prefix.tpe}
    \only<2->{val ConstantType(Constant(sUrl: String)) = tUrl}
    \only<2->{val schema = loadSchema(sUrl)}
    \only<3->{val Literal(Constant(sField: String)) = field}
    \only<3->{if (schema.contains(sField)) q"\$\{c.prefix\}(\$sField)"}
    \only<3->{else c.abort(s"value \$sField is not a member of \$sUrl")}
  \}
\}
\end{semiverbatim}
\end{frame}

\begin{frame}[fragile]
\frametitle{}

\vskip40pt
\begin{center}
\text{\color{blue}{\Large{Materialization of type class instances}}}
\end{center}
\end{frame}

\begin{frame}[fragile]
\frametitle{Type classes as objects and implicits}

\begin{semiverbatim}
trait Showable[T] \{ def show(x: T): String \}
def show[T](x: T)(implicit s: Showable[T]) = s show x

implicit object IntShowable \{
  def show(x: Int) = x.toString
\}

show(42) // "42"
show("42") // compilation error

\end{semiverbatim}
\begin{itemize}
\item In Scala type classes can be modelled with traits
\item Type class instances are modelled with implicit values
\item Implicit search automates type-driven selection of instances
\end{itemize}
\end{frame}

\begin{frame}[fragile]
\frametitle{Boilerplate}

\begin{semiverbatim}
class C(x: Int)
implicit def cShowable = new Showable[C] \{
  def show(c: C) = "C(" + c.x + ")"
\}

class D(x: Int)
implicit def dShowable = new Showable[D] \{
  def show(d: D) = "D(" + d.x + ")"
\}

\end{semiverbatim}
\begin{itemize}
\item Instance definitions for similar types are frequently very similar
\item Leads to proliferation of boilerplate code
\item There are techniques to abstract boilerplate, but they have downsides
\end{itemize}
\end{frame}

\begin{frame}[fragile]
\frametitle{Materialization}

\begin{semiverbatim}
trait Showable[T] \{ def show(x: T): String \}

object Showable \{
  implicit def materialize[T]: Showable[T] = macro ...
\}

\end{semiverbatim}
\begin{itemize}
\item Implicit scope of \texttt{Showable} includes the members of its companion
\item Failed searches for \texttt{Showable[T]} fall back to \texttt{materialize[T]}
\item \texttt{materialize[T]} analyzes \texttt{T} and generates an appropriate instance
\end{itemize}
\end{frame}

\begin{frame}[fragile]
\frametitle{Synergy}

\begin{semiverbatim}
implicit def listShowable[T](implicit s: Showable[T]) =
  new Showable[List[T]] \{
    def show(x: List[T]) = \{
      x.map(s.show).mkString("List(", ", ", ")")
    \}
  \}

show(List(42))
// prints: List(42)

\end{semiverbatim}

\begin{itemize}
\item Materialization seamlessly melds into vanilla implicit search
\item In the example above, \texttt{Showable[List[Int]]} will be provided by \texttt{listShowable}
\item Building up on a materialized instance of \texttt{Showable[Int]}
\end{itemize}
\end{frame}

\begin{frame}[fragile]
\frametitle{Comparison with generic programming}

\begin{itemize}
\item Materialization provides performance of manually written code
\item Generic programming is arguably easier to use (no tree manipulation)
\item Is it possible to combine the strong points of both approaches?
\end{itemize}
\end{frame}

\begin{frame}[fragile]
\frametitle{}

\vskip40pt
\begin{center}
\text{\color{blue}{\Large{Summary}}}
\end{center}
\end{frame}

\begin{frame}[fragile]
\frametitle{Summary}

\begin{itemize}
\item Scala Macros, macros that enjoy rich syntax and static types
\item Seamlessly integrate with vanilla Scala features
\item Empower existing features with code generation and compile-time programmability capabilities
\item Work well in a production version of Scala for a number of case studies
\end{itemize}
\end{frame}

\begin{frame}[fragile]
\frametitle{Future work}

\begin{itemize}
\item Formalize the design of macros in a flexible and tractable framework
\item Explore compile-time techniques that don't fit into textual abstraction
\end{itemize}
\end{frame}

\end{document}
