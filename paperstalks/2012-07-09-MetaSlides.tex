\documentclass[hyperref={bookmarks=false}]{beamer}

\useoutertheme{infolines}
\setbeamertemplate{headline}{} % removes the headline that infolines inserts
% \setbeamertemplate{footline}{
%   \hfill%
%   \usebeamercolor[fg]{page number in head/foot}%
%   \usebeamerfont{page number in head/foot}%
%   \insertpagenumber\,/\,\insertpresentationendpage\kern1em\vskip2pt%
% }
\setbeamertemplate{footline}{
  \hfill%
  \usebeamercolor[fg]{page number in head/foot}%
  \usebeamerfont{page number in head/foot}%
  \insertpagenumber\kern1em\vskip2pt%
}
\setbeamertemplate{navigation symbols}{}

\usepackage[parfill]{parskip}
\usepackage{color}
\usepackage{listings}
\usepackage{hyperref}
\hypersetup{pdfauthor={Eugene Burmako},pdfsubject={Scala Macros},pdftitle={Scala Macros}}
\title{Scala Macros}

\definecolor{linkblue}{RGB}{49,57,174}
\definecolor{dkgreen}{rgb}{0,0.6,0}
\definecolor{gray}{rgb}{0.5,0.5,0.5}
\definecolor{mauve}{rgb}{0.58,0,0.82}

\lstdefinelanguage{scala}{
  morekeywords={abstract,annotation,case,catch,class,def,%
    do,else,extends,false,final,finally,%
    for,if,implicit,import,match,mixin,%
    new,null,object,override,package,%
    private,protected,requires,return,sealed,%
    super,this,throw,trait,true,try,%
    type,val,var,while,with,yield,
    macro},
  sensitive=true,
  morecomment=[l]{//},
  morecomment=[n]{/*}{*/}
%  morestring=[b]",
%  morestring=[b]',
%  morestring=[b]"""
}

\lstset{frame=tb,
  language=scala,
  aboveskip=3mm,
  belowskip=3mm,
  showstringspaces=false,
  columns=flexible,
  basicstyle={\small\ttfamily},
  numbers=none,
  numberstyle=\tiny\color{gray},
  keywordstyle=\color{blue},
  commentstyle=\color{dkgreen},
  stringstyle=\color{mauve},
  frame=single,
  breaklines=true,
  breakatwhitespace=true
  tabsize=3
}

\AtBeginSection[]
{
  \begin{frame}
    \frametitle{Outline}
    \tableofcontents[currentsection]
  \end{frame}
}

\begin{document}

\title{Scala Macros}
\author{Eugene Burmako \and Martin Odersky}
\institute{\'Ecole Polytechnique F\'ed\'erale de Lausanne}
\date{09 July 2012 / Meta 2012}
{
\setbeamertemplate{footline}{}
\begin{frame}
  \titlepage
\end{frame}
}

\begin{frame}[fragile]
\frametitle{What is this talk about?}
\begin{itemize}
\item Compile-time metaprogramming
\item Type-safe AST transformers (called "macros" in Scala and in several other languages)
\item Road to macros in Scala
\end{itemize}
\end{frame}

\begin{frame}[fragile]
\frametitle{Behind the scenes}
\begin{itemize}
\item Advanced features of Scala's type system w.r.t macros
\begin{itemize}
\item Cross-stage path-dependent types
\item Type inference in presence of macros
\item Implicits in macro declarations and implementations
\end{itemize}
\item Design of the Scala reflection library
\begin{itemize}
\item Cake pattern to provide different views into the compiler
\item Abstract types that enable virtual classes
\item Uniformity of compile-time and runtime reflection
\end{itemize}
\end{itemize}
\end{frame}

\section{Introduction}

\begin{frame}[fragile]
\frametitle{Project Kepler}

The project was started in October 2011 with the following goals in mind:
\begin{itemize}
\item To democratize metaprogramming (at the moment there's a lot of hype that the future is multicore; along the similar lines my belief is that the future is meta).
\item To solve several hot problems in Scala: insufficient control over inlining, need for reification in domain-specific languages.
\end{itemize}

Since April 2012 (milestone pre-release 2.10.0-M3) macros are a part of Scala. Several practical (data access facility, unit testing framework, library for numeric computations) and research projects are already using macros.
\end{frame}

\begin{frame}[fragile]
\frametitle{Macros in Scala}
\begin{lstlisting}[language=scala]
def assert(cond: Boolean, msg: Any) = macro impl
def impl(c: Context)(cond: c.Expr[Boolean], msg: c.Expr[Any]) =
  if (assertionsEnabled)
    // if (!cond) raise(msg)
    If(Select(cond.tree, newTermName("$unary_bang")), Apply(Ident(newTermName("raise")), List(msg.tree)), Literal(Constant(())))
  else
    Literal(Constant(())

import assert
assert(2 + 2 == 4, "weird arithmetic")
\end{lstlisting}

\begin{itemize}
\item Metalanguage = target language
\item Macros work with ASTs rather than with text
\item Type awareness and type-safety
\end{itemize}
\end{frame}

\begin{frame}[fragile]
\frametitle{Putting macros in perspective}
\begin{itemize}
\item Text generators (C/C++ preprocessor, M4). No integration into grammar or semantics of the target language.
\item Syntax extenders (CamlP4, SugarHaskell, Marco). Define new productions and non-terminals for the original grammar. Rely on ad-hoc tricks to get semantic information (bindings, types, etc).
\item Deeply embedded DSLs (Virtualized Scala, LMS). Reuse host parser and typechecker, yet can override semantics.
\item Macros (LISP, Scheme, MacroML, Template Haskell, \emph{Nemerle}, etc). Integrated into the compiler, expand during compilation, typically have access to the compiler API.
\item Metalanguages (N2). Every aspect of a language (parser, type checker, code generation, IDE integration) is customizable.
\end{itemize}
\end{frame}

\begin{frame}[fragile]
\frametitle{Why this talk is interesting}
\begin{itemize}
\item Metacomputations (because metaprogramming is cool)
\item Linguistics (several notational problems and solutions w.r.t metaprogramming)
\item Metamacros (macro-generating macros, notation macros, self-cleaning macros)
\end{itemize}
\end{frame}

\section{Notation}

\begin{frame}[fragile]
\frametitle{Notation for macro triggers}

Function application (very traditional, implemented):
\begin{lstlisting}[language=scala]
macro def assert(cond: Boolean, msg: Any) = ...
assert(2 + 2 == 4, "weird arithmetic")
\end{lstlisting}

Type construction (a natural desire for a typed language, planned):
\begin{lstlisting}[language=scala]
macro type MySqlDb(connString: String) = ...
type MyDb = Base with MySqlDb("Server=127.0.0.1")
\end{lstlisting}

Declaration of program elements (tentative):
\begin{lstlisting}[language=scala]
macro annotation Serializable = ...
@Serializable class Person(...)
\end{lstlisting}

Choice of macro triggers is arbitrary and ad-hoc. To do better we need integration with the parser (Nemerle, SugarHaskell).
\end{frame}

\begin{frame}[fragile]
\frametitle{Notation for metacode}
The first approach (similar to Template Haskell and Nemerle):
\begin{lstlisting}[language=scala]
macro def assert(cond: Boolean, msg: Any) =
  if (assertionsEnabled)
    If(Select(cond, newTermName("$unary_bang")), Apply(Ident(newTermName("raise")), List(msg)), Literal(Constant(())))
  else
    Literal(Constant(())
\end{lstlisting}

\begin{itemize}
\item Minimalistic and appealing at a glance
\item Transparent to the user, as the signature doesn't reveal the underlying magic
\item Cross-stage lexical scoping is very potent
\item Too potent to be robust
\end{itemize}
\end{frame}

\begin{frame}[fragile]
\frametitle{Problem with the first approach}
\begin{lstlisting}[language=scala]
class Queryable[T, Repr](query: Query) {
  macro def filter(p: T => Boolean): Repr =
    Apply(Ident(newTermName("Query")),
      List(Apply(Ident(newTermName("Filter")),
        List(query, reify(p)))))
}
\end{lstlisting}

\begin{itemize}
\item \emph{p} being used in a macro expansion is okay, since it comes from the same metalevel.
\item But what about \emph{query}? This is a runtime value, so we cannot splice it into the result of macro expansion.
\item Adapting closure conversion we could make it work, but that would bring significant technical and cognitive problems.
\item To avoid this problem Template Haskell and Nemerle only allow macros in top-level definitions!
\end{itemize}
\end{frame}

\begin{frame}[fragile]
\frametitle{Notation for metacode}

The second approach:
\begin{lstlisting}[language=scala]
def assert(cond: Boolean, msg: Any) = macro impl
def impl(c: Context)(cond: c.Expr[Boolean], msg: c.Expr[Any]) =
  if (assertionsEnabled)
    If(Select(cond.tree, newTermName("$unary_bang")), Apply(Ident(newTermName("raise")), List(msg.tree)), Literal(Constant(())))
  else
    Literal(Constant(())
\end{lstlisting}

\begin{itemize}
\item Splits macro definitions and macro implementations. The latter are only allowed in static contexts.
\item As a pleasant side effect, macro parameter magic is gone, and macro implementations are now first-class.
\end{itemize}
\end{frame}

\begin{frame}[fragile]
\frametitle{Notation for quasiquoting}

An obvious approach is to introduce new syntax, following multiple languages which have done that:

\begin{lstlisting}[language=scala]
<[ if (!$cond) raise($msg) ]>
\end{lstlisting}

Being obvious this design decision is also suboptimal:
\begin{itemize}
\item Adds extra burden on the language spec
\item Complicates parsing
\item Is opaque to existing tools
\end{itemize}
\end{frame}

\begin{frame}[fragile]
\frametitle{Notation for quasiquoting}

When macros started brewing, Scala got string interpolation. We generalized the interpolation proposal to accomodate a wide range of syntaxes:

\begin{lstlisting}[language=scala]
scala"if (!$cond) raise($msg)"
\end{lstlisting}

gets desugared by the parser into the following snippet:

\begin{lstlisting}[language=scala]
StringContext("if (!", ") raise (", ")").scala(cond, msg)
\end{lstlisting}

\begin{itemize}
\item No changes to the compiler
\item Modularity and extensibility (anyone can "pimp" the scala method onto StringContext with implicit conversions)
\item Partial amenability to automatic analysis
\end{itemize}
\end{frame}

\section{Reification}

\begin{frame}[fragile]
\frametitle{A discovery}

Macro-based string interpolation expressing quasiquotes is nice, being minimalistic, expressive and performant. What's even more important, it's conventional.

\begin{lstlisting}[language=scala]
scala"if (!$cond) raise($msg)"
\end{lstlisting}

A key insight however was to explore the design space further, which gave us this marvel:

\begin{lstlisting}[language=scala]
reify(if (!cond.eval) raise(msg.eval))
\end{lstlisting}
\end{frame}

\begin{frame}[fragile]
\frametitle{Notation for quasiquoting}

The final solution:

\begin{lstlisting}[language=scala]
reify(if (!cond.eval) raise(msg.eval))
\end{lstlisting}

\emph{reify} is a library macro (but could be implemented by a programmer).

It takes an AST that represents an expression (which in Scala can be even a declaration or a sequence of declarations) and generates a tree that will re-create that AST at runtime.

\emph{eval} method is a marker that tells \emph{reify} to splice the target expression into the resulting AST.
\end{frame}

\begin{frame}[fragile]
\frametitle{Hygiene}
\begin{lstlisting}[language=scala]
def raise(msg: Any) = throw new AssertionError(msg)

def assert(cond: Boolean, msg: Any) = macro impl
def impl(c: Context)(cond: c.Expr[Boolean], msg: c.Expr[Any]) =
  c.reify(if (!cond.eval) raise(msg))

object Test extends App {
  def raise(msg: Any) = { /* haha, tricked you */ }
  assert(2 + 2 == 3, "no way")
}
\end{lstlisting}

\begin{itemize}
\item \emph{reify} solves the problem of inadvertent name captures
\item For example \emph{raise} in the AST produced will bind to the original \emph{raise} no matter where it ends up after macro expansion
\end{itemize}
\end{frame}

\begin{frame}[fragile]
\frametitle{Staging}

\emph{reify} can also express staging:

\begin{itemize}
\item Brackets are implemented by \emph{reify} itself
\item Escape is implemented inside \emph{reify} by treating \emph{eval} functions in a special way
\item Run is implicitly carried out by compilation and macro expansion (for nested \emph{reify} calls)
\end{itemize}

\end{frame}

\begin{frame}[fragile]
\frametitle{Related work}

Taha et al. build a macro system atop a staged language:
\begin{itemize}
\item \emph{Macros as Multi-Stage Computations} Ganz, Sabry \& Taha, ICFP'01
\item \emph{Staged Notational Definitions} Taha \& Johann, GPCE'03
\end{itemize}

We build a staged system atop a macro language.

\end{frame}

\begin{frame}[fragile]
\frametitle{Reified types}

\emph{reify} saves syntax trees and transfers them to the next metalevel. Exactly the same can be done for types!

With \emph{reify} it becomes possible to inspect type arguments of polymorphic functions and type constructors.
\end{frame}

\section{Wrapping up}

\begin{frame}[fragile]
\frametitle{Summary}

\begin{itemize}
\item Language = metalanguage
\item Macro-enabled Scala = Scala + \emph{macro} keyword + a trigger in typer
\item This minimalistic core is enough to express \emph{reify}, which can implement quasiquoting, hygiene and staging
\item Reification makes Scala homoiconic
\item This is officially a part of Scala
\end{itemize}
\end{frame}

\end{document}
