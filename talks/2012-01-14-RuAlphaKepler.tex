\documentclass[hyperref={bookmarks=false}]{beamer}

\usepackage[utf8]{inputenc}
\usepackage[english,russian]{babel}
\usepackage[parfill]{parskip}

\usepackage{color}
\usepackage{listings}
\usepackage{hyperref}
\hypersetup{pdfauthor={Eugene Burmako},pdfsubject={Project Kepler},pdftitle={Alpha Kepler}}
\title{Alpha Kepler}

\definecolor{linkblue}{RGB}{49,57,174}
\definecolor{dkgreen}{rgb}{0,0.6,0}
\definecolor{gray}{rgb}{0.5,0.5,0.5}
\definecolor{mauve}{rgb}{0.58,0,0.82}

\lstdefinelanguage{scala}{
  morekeywords={abstract,annotation,case,catch,class,def,%
    do,else,extends,false,final,finally,%
    for,if,implicit,import,match,mixin,%
    new,null,object,override,package,%
    private,protected,requires,return,sealed,%
    super,this,throw,trait,true,try,%
    type,val,var,while,with,yield,
    macro},
  sensitive=true,
  morecomment=[l]{//},
  morecomment=[n]{/*}{*/},
  morestring=[b]",
  morestring=[b]',
  morestring=[b]"""
}

\lstset{frame=tb,
  language=scala,
  aboveskip=3mm,
  belowskip=3mm,
  showstringspaces=false,
  columns=flexible,
  basicstyle={\small\ttfamily},
  numbers=none,
  numberstyle=\tiny\color{gray},
  keywordstyle=\color{blue},
  commentstyle=\color{dkgreen},
  stringstyle=\color{mauve},
  frame=single,
  breaklines=true,
  breakatwhitespace=true
  tabsize=3
}

\AtBeginSection[]
{
  \begin{frame}
    \frametitle{План}
    \tableofcontents[currentsection]
  \end{frame}
}

\begin{document}

\title{$\alpha$-Кеплер}
\author{Евгений Бурмако}
\institute{EPFL, LAMP}
\date{14 января 2012}
\maketitle

\begin{frame}[t,fragile]
\frametitle{$\alpha$-Кеплер}

Всем привет! Меня зовут Евгений Бурмако, с прошлой осени я работаю в Scala Team и параллельно учусь в аспирантуре EPFL на кафедре Мартина Одерского.

Сегодня мы обсудим прогресс проекта ``Кеплер'', в рамках которого реализуются \textbf{макросы} и \textbf{квазицитаты} - средства метапрограммирования времени компиляции для Скалы. В процессе общения мы реализуем одну штуку, о которой я мечтал со времен знакомства с Немерле. Также мы поговорим о том, какие незапланированные применения макросов нашлись в процессе работы над проектом.

В \text{\color{linkblue}\href{http://scalamacros.org/news/2011-10-29-talk-at-scalaby-meetup.html}{предыдущем выступлении}} я рассматривал макросы с теоретической точки зрения, а сегодня будет, в основном, практика. Поэтому перед тем, как продолжить чтение, может быть полезно просмотреть {\color{linkblue}\href{http://scalamacros.org/talks/2011-10-29-RuProjectKepler.pdf}{слайды прошлого рассказа}}.
\end{frame}

\section{Введение в макросы}

\begin{frame}[t,fragile]
\frametitle{Начнем издалека}

Одним из интересных событий этой осени в мире Скалы стало обсуждение \text{\color{linkblue}\href{https://docs.google.com/document/d/1NdxNxZYodPA-c4MLr33KzwzKFkzm9iW9POexT9PkJsU/edit?hl=en_US&pli=1}{интерполяции строк}}.

Эта функциональность встречается во многих скриптовых языках (bash, Perl, Ruby) и предоставляет возможность встраивать в строки переменные из лексического окружения.

\begin{lstlisting}[language=scala]
scala> val world = "world"
world: String = world

scala> println(s"hello ${world}!")
hello world!
\end{lstlisting}%$

Буква \texttt{s}, стоящая прямо перед строкой - не опечатка, а обозначение того, что строка интерполируется (в целях обратной совместимости доллары в обычных строках не будут интерпретироваться специальным образом).
\end{frame}

\begin{frame}[t,fragile]
\frametitle{Реализация интерполятора}

Несмотря на свою простоту, интерполяция не поддается библиотечной реализации, так как она оперирует лексическим окружением, которое в явном виде недоступно.

В рамках смелого научного эксперимента нам придется открыть \texttt{doTypedApply} (функцию, которая типизирует применение методов) и вставить проверку на метод интерполяции. Внутри компилятора у нас есть полный доступ ко всей семантической информации о программе, чем мы и воспользуемся.

В тайпере возможно получить строковый литерал, который требуется проинтерполировать, и прямо на месте распарсить этот литерал. После этого, пройдясь по дереву контекстов, несложно составить словарь видимых переменных и превратить вызов метода \texttt{s} в обычную конкатенацию строк.

Если очень интересно, детали реализации можно посмотреть \text{\color{linkblue}\href{https://github.com/scalamacros/kepler/blob/09d1aed7e2353d93aaf703bbae83c1f4322cd450/src/compiler/scala/tools/nsc/typechecker/Quasiquoter.scala\#L94}{у меня в репозитории}}, а пока что пойдем дальше.
\end{frame}

\begin{frame}[t,fragile]
\frametitle{Использование интерполятора}

Пока что все было довольно несложно. Мы встроились в тайпер и заменили вызов функции-маркера на вручную собранное дерево. Звучит страшнее, чем оно есть на самом деле. Через несколько слайдов мы будем заниматься тем же самым в режиме лайв =)

Единственный вопрос в том, как использовать наш патч к компилятору. Интерполяция выглядит более-менее серьезно, но вот трюки, специфичные для проекта, в апстрим явно не примут, а таскать за собой кастомную сборку компилятора неудобно по многим причинам.

Общепринятым решением в данной ситуации является написание \text{\color{linkblue}\href{http://www.scala-lang.org/node/140}{плагина к компилятору}} (например, CPS в Скале реализован именно через плагин), а это очень близко к философии макросов. Мы почти на месте.
\end{frame}

\begin{frame}[t,fragile]
\frametitle{Что такое макросы?}

Макросы - специального вида плагины к компилятору, которые автоматически загружаются из classpath и преобразуют заданные элементы программы (вызовы функций, ссылки на типы, объявления классов и методов).

По аналогии с рассмотренным выше примером макросы:
\begin{itemize}
\item Выполняются во время компиляции
\item В отличие от традиционных плагинов прозрачно загружаются компилятором, не требуя оборачивания в \texttt{Plugin} и \texttt{Component}
\item Работают с деревьями выражений, которые соответствуют коду компилируемой программы
\item Имеют доступ к внутренним сервисам компилятора (рефлексия ранее скомпилированного кода, вывод и проверка типов, лексические окружения и так далее)
\end{itemize}
\end{frame}


\begin{frame}[t,fragile]
\frametitle{Какие бывают макросы?}

Макро-функции получают на вход AST аргументов, раскрываются в AST и инлайнятся в точку вызова.

\begin{lstlisting}[language=scala]
macro def printf(format: String, params: Any*) = ...
printf("Value = %d", 123 + 877)
\end{lstlisting}

Макро-типы представляют собой типы, члены которых генерируются во время компиляции:

\begin{lstlisting}[language=scala]
macro class MySqlDb(connString: String) = ...
object MyDb extends MySqlDb("Database=Foo;")
\end{lstlisting}

Макро-аннотации выполняют пост-обработку объявлений методов и типов:

\begin{lstlisting}[language=scala]
macro annotation Serializable(implicit ctx: Context) = ...
@Serializable case class Person(name: String)
\end{lstlisting}
\end{frame}

\section{Статус проекта ``Кеплер''}

\begin{frame}[t,fragile]
\frametitle{Отклик}

За несколько месяцев жизни проекта ``Кеплер'' нашлось немало практических задач, которые сильно упрощаются при наличии макросов.

Уже есть желающие использовать макросы для реализации языка запросов в O/RM, я слышал о планах использовать макро-аннотации в линзах, есть идеи насчет применения макро-типов для генерации бойлерплейта, необходимого для вычислений на типах. Буквально позавчера Мартин \text{\color{linkblue}\href{https://github.com/odersky/scala/commit/35b36229b189a840756330049fa1c7094a309036}{реализовал прототип оптимизатора}}, который использует макросы для ускорения \texttt{Range.foreach}.

Наша лаборатория выиграла грант \text{\color{linkblue}\href{http://www.kti.admin.ch/index.html?lang=en}{Швейцарской комиссии по технологиям и инновациям}} на разработку усовершенствования Скалы, в основе которого лежат рефлексия и макросы.
\end{frame}

\begin{frame}[t,fragile]
\frametitle{Статус}

На сегодняшний момент мы переосмыслили публичный интерфейс к компилятору и его структурам данных и воплотили в жизнь \text{\color{linkblue}\href{https://github.com/scalamacros/kepler/branches/topic/macros}{прототип компилятора}}, который реализует макро-функции. Мы довольны тем, как выглядит прототип, и будем дальше двигаться в этом направлении.

В активной разработке находятся \text{\color{linkblue}\href{http://xeno-by.livejournal.com/69111.html}{квазицитаты}} - доменно-специфический язык для создания и декомпозиции абстрактных синтаксических деревьев. Скорее всего, они будут реализованы как обобщение \text{\color{linkblue}\href{https://docs.google.com/document/d/1NdxNxZYodPA-c4MLr33KzwzKFkzm9iW9POexT9PkJsU/edit?hl=en_US&pli=1}{строковой интерполяции}}, но здесь гораздо меньше ясности, чем с макросами.

Мы опубликовали сайт \text{\color{linkblue}\href{http://scalamacros.org}{scalamacros.org}}, на котором собраны материалы по нашему проекту. Особый интерес, на наш взгляд, составляет раздел \text{\color{linkblue}\href{http://scalamacros.org/usecases.html}{``Use Cases''}}, в котором рассматриваются области применения макросов.
\end{frame}

\begin{frame}[t,fragile]
\frametitle{Что дальше?}

Наш следующий шаг - стабилизация рефлексии (которая используется макросами для доступа к деревьям и компилятору) и выпуск бета-версии макросов.

Вместе с бета-версией мы опубликуем документы в рамках Scala Improvement Process, в которых предложим включить макросы и, возможно, квазицитаты в версию 2.10 (которая выйдет в первой половине года).

За новостями можно следить по следующим направлениям: \text{\color{linkblue}\href{http://scalamacros.org/news.html}{новости на scalamacros.org}} (официальные объявления), \text{\color{linkblue}\href{http://xeno-by.livejournal.com/tag/macros2011}{посты в моем ЖЖ}} (дизайн-заметки и майлстоуны), \text{\color{linkblue}\href{http://www.twitter.com/\#!/xeno_by}{мессаги в моем твиттере}} (с этим все понятно :)).
\end{frame}

\section{Макрос для регулярных выражений}

\begin{frame}
\frametitle{Демонстрация}
(В оффлайновой версии слайдов тут будет транскрипт нашей сессии кодинга)
\end{frame}

\section{Заключение}

\begin{frame}[t]
\frametitle{Резюме}
\begin{itemize}
\item Механизм макросов предоставляет прозрачные точки расширения компилятора. С помощью макросов можно расширять семантику Скалы - анализировать и оптимизировать код, генерировать бойлерплейт и \text{\color{linkblue}\href{http://scalamacros.org/usecases.html}{многое другое}}.
\item Макросы в Скале - это уже реальность. В ближайшем будущем мы выпустим бета-версию и представим пропоузал по добавлению макросов в 2.10.
\item По шагам нашей демонстрации уже сейчас можно поэкспериментировать со своими собственными макросами. Если что-то не получится - напишите мне на \text{\color{linkblue}\href{mailto:eugene.burmako@epfl.ch}{eugene.burmako@epfl.ch}}, разберемся вместе.
\end{itemize}
\end{frame}

\begin{frame}[t]
\frametitle{Ссылки}
\begin{itemize}

\item Project Kepler, Compile-Time Metaprogramming for Scala\\
\text{\color{linkblue}\href{https://github.com/scalamacros/kepler}{https://github.com/scalamacros/kepler}}

\item Официальный сайт, документация и примеры\\
\text{\color{linkblue}\href{http://scalamacros.org/}{http://scalamacros.org}}

\item Живой журнал на русском, обновляется чаще всего\\
\text{\color{linkblue}\href{http://xeno-by.livejournal.com/tag/macros2011}{http://xeno-by.livejournal.com/tag/macros2011}}

\end{itemize}
\end{frame}

\begin{frame}[c, fragile]
\frametitle{}

\centering
{\Large Вопросы и ответы}\\

\centering
\text{\color{linkblue}\href{mailto:eugene.burmako@epfl.ch}{eugene.burmako@epfl.ch}}

\end{frame}

\end{document}
