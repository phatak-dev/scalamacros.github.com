\documentclass[hyperref={bookmarks=false}]{beamer}

\usepackage[parfill]{parskip}
\usepackage{color}
\usepackage{listings}
\usepackage{hyperref}
\hypersetup{pdfauthor={Eugene Burmako},pdfsubject={Project Kepler},pdftitle={Project Kepler}}
\title{Project Kepler}

\definecolor{linkblue}{RGB}{49,57,174}
\definecolor{dkgreen}{rgb}{0,0.6,0}
\definecolor{gray}{rgb}{0.5,0.5,0.5}
\definecolor{mauve}{rgb}{0.58,0,0.82}

\lstdefinelanguage{scala}{
  morekeywords={abstract,annotation,case,catch,class,def,%
    do,else,extends,false,final,finally,%
    for,if,implicit,import,match,mixin,%
    new,null,object,override,package,%
    private,protected,requires,return,sealed,%
    super,this,throw,trait,true,try,%
    type,val,var,while,with,yield,
    macro},
  sensitive=true,
  morecomment=[l]{//},
  morecomment=[n]{/*}{*/}
%  morestring=[b]",
%  morestring=[b]',
%  morestring=[b]"""
}

\lstset{frame=tb,
  language=scala,
  aboveskip=3mm,
  belowskip=3mm,
  showstringspaces=false,
  columns=flexible,
  basicstyle={\small\ttfamily},
  numbers=none,
  numberstyle=\tiny\color{gray},
  keywordstyle=\color{blue},
  commentstyle=\color{dkgreen},
  stringstyle=\color{mauve},
  frame=single,
  breaklines=true,
  breakatwhitespace=true
  tabsize=3
}

\begin{document}

\title{Project Kepler}
\author{Eugene Burmako}
\institute{EPFL, LAMP}
\date{17 January 2012}
\maketitle

\begin{frame}[fragile]
\frametitle{What?}

My research is about compile-time metaprogramming, i.e. about answering the question:
``how to empower developers so that they can extend the compiler, but stay sane and not mess it up?''
\end{frame}

\begin{frame}[fragile]
\frametitle{Why?}

CTM enables the following techniques:
\begin{itemize}
\item language virtualization
\item program reification
\item self-optimization
\item algorithmic program construction
\end{itemize}
\end{frame}

\begin{frame}[fragile]
\frametitle{How?}

Macros and quasiquotes. The former make extending the compiler possible, the latter make it bearable.
\end{frame}

\begin{frame}[fragile]
\frametitle{But!}

Q: Scala has enough advanced features for its creator to think about introducing feature switches.
Why bother?

A: CTM lets us advance in several areas that are hot for the community:
code lifting for better DSLs,
domain-specific optimization for high performance,
(speculation) type-level computations for principled type hackery.
\end{frame}

\begin{frame}[fragile]
\frametitle{Zen}

\begin{lstlisting}[language=scala]
class Queryable[T, Repr](query: Query) {
  macro def filter(p: T => Boolean): Repr = scala"""
    val b = $newBuilder
    b.query = Filter($query, ${reify(p)})
    b.result
  """
}
\end{lstlisting}
\end{frame}

\begin{frame}[fragile]
\frametitle{Now}

Prototypes: \text{\color{linkblue}\href{http://github.com/scalamacros/kepler}{http://github.com/scalamacros/kepler}}
(macro defs, quasiquotes, splicing, pattern matching)

Documentation: \text{\color{linkblue}\href{http://scalamacros.org}{http://scalamacros.org}}
(use cases, talks and walkthroughs, alpha versions of proposals)
\end{frame}

\begin{frame}[fragile]
\frametitle{Use}

\begin{itemize}
\item Slick language integrated connector kit
\item Lenses
\item Shapeless
\item Domain-specific inlining
\end{itemize}
\end{frame}

\begin{frame}[fragile]
\frametitle{Next?}

SIP, stabilization, macro types and macro annotations.
\end{frame}

\begin{frame}[fragile]
\frametitle{Thanks!}

\centering
\text{\color{linkblue}\href{mailto:eugene.burmako@epfl.ch}{eugene.burmako@epfl.ch}}

\end{frame}

\end{document}
