\documentclass[hyperref={bookmarks=false}]{beamer}

\usepackage[parfill]{parskip}
\usepackage{color}
\usepackage{listings}
\usepackage{hyperref}
\hypersetup{pdfauthor={Eugene Burmako},pdfsubject={Metaprogramming in Scala 2.10},pdftitle={Metaprogramming in Scala 2.10}}
\title{Metaprogramming in Scala 2.10}

\definecolor{linkblue}{RGB}{49,57,174}
\definecolor{dkgreen}{rgb}{0,0.6,0}
\definecolor{gray}{rgb}{0.5,0.5,0.5}
\definecolor{mauve}{rgb}{0.58,0,0.82}

\lstdefinelanguage{scala}{
  morekeywords={abstract,annotation,case,catch,class,def,%
    do,else,extends,false,final,finally,%
    for,if,implicit,import,match,mixin,%
    new,null,object,override,package,%
    private,protected,requires,return,sealed,%
    super,this,throw,trait,true,try,%
    type,val,var,while,with,yield,
    macro},
  sensitive=true,
  morecomment=[l]{//},
  morecomment=[n]{/*}{*/}
%  morestring=[b]",
%  morestring=[b]',
%  morestring=[b]"""
}

\lstset{frame=tb,
  language=scala,
  aboveskip=3mm,
  belowskip=3mm,
  showstringspaces=false,
  columns=flexible,
  basicstyle={\small\ttfamily},
  numbers=none,
  numberstyle=\tiny\color{gray},
  keywordstyle=\color{blue},
  commentstyle=\color{dkgreen},
  stringstyle=\color{mauve},
  frame=single,
  breaklines=true,
  breakatwhitespace=true
  tabsize=3
}

\AtBeginSection[]
{
  \begin{frame}
    \frametitle{Agenda}
    \tableofcontents[currentsection]
  \end{frame}
}

\begin{document}

\title{Metaprogramming in Scala 2.10}
\author{Eugene Burmako}
\institute{EPFL, LAMP}
\date{28 April 2012}
\maketitle

\section{Intro}

\begin{frame}[fragile]
\frametitle{Metaprogramming}

\begin{quote}
Metaprogramming is the writing of computer programs that write or manipulate other programs or themselves as their data.
\end{quote}

\begin{flushright}
\textemdash Wikipedia
\end{flushright}

\end{frame}

\begin{frame}[fragile]
\frametitle{Compiler}

Q: How to enable metaprogramming?

A: Who has more data about a program than a compiler?

Let's expose the compiler to the programmer.

\end{frame}

\begin{frame}[fragile]
\frametitle{Reflection}

In 2.10 we expose data about programs via reflection API.

The API is spread between scala-library.jar (interfaces), scala-reflect.jar (implementations) and scala-compiler.jar \\ (runtime compilation).

\end{frame}

\begin{frame}[fragile]
\frametitle{Let Martin speak}

Design of the reflection API is explained in great detail in a recent talk by Martin Odersky:

\text{\color{linkblue}\href{http://channel9.msdn.com/Events/Lang-NEXT/Lang-NEXT-2012/Reflection-and-Compilers}
{channel9.msdn.com/Lang-NEXT-2012/Reflection-and-Compilers}}

\end{frame}

\begin{frame}[fragile]
\frametitle{Hands on}

Today we will learn the fundamentals of reflection API and learn to learn more about it
via a series of hands-on examples.

\end{frame}

\begin{frame}[fragile]
\frametitle{Macros}

Q: Hey! What about macros?

A: Reflection is at the core of macros, reflection provides macros with an API, reflection enables macros.
Our focus today is understanding reflection, macros are just a tiny bolt-on.

For more information about macros, their philosophy and applications, take a look at my Scala Days talk:
\text{\color{linkblue}\href{http://scalamacros.org/talks/2012-04-18-ScalaDays2012.pdf}
{http://scalamacros.org/talks/2012-04-18-ScalaDays2012.pdf}}.

\end{frame}

\section{Reflection}

\begin{frame}[fragile]
\frametitle{Core data structures}

\begin{itemize}
\item Trees
\item Symbols
\item Types
\end{itemize}

\begin{lstlisting}[language=XML]
C:\Projects\Kepler>scalac -Xshow-phases
phase name  id  description
----------  --  -----------
    parser   1  parse source into ASTs, simple desugaring
     namer   2  resolve names, attach symbols to named trees
     typer   4  the meat and potatoes: type the trees
   pickler   7  serialize symbol tables
\end{lstlisting}

I'll do my best to explain these concepts, but it's barely possible to do it better than Paul Phillips.
Be absolutely sure to watch the \text{\color{linkblue}\href{http://skillsmatter.com/podcast/scala/scalac-internals}
{Inside the Sausage Factory}} talk (when the video is up).
\end{frame}

\begin{frame}[fragile]
\frametitle{Trees}

Short-lived, mostly immutable, mostly plain case classes.

\begin{lstlisting}[language=Scala]
Apply(Ident("println"), List(Literal(Constant("hi!"))))
\end{lstlisting}

Comprehensive list of public trees can be found here:
\text{\color{linkblue}\href{https://github.com/scala/scala/blob/master/src/library/scala/reflect/api/Trees.scala}
{scala/reflect/api/Trees.scala}}
(be sure to take a look at the "standard pattern match" section in the end of the file).

\end{frame}

\begin{frame}[fragile]
\frametitle{Learn to learn}

\begin{itemize}
\item -Xprint:parser
\item -Yshow-trees and its cousins
\item scala.reflect.mirror.showRaw(scala.reflect.mirror.reify(...))
\end{itemize}

Q: Where do I pull these compiler flags from?

A: \text{\color{linkblue}\href{https://github.com/scala/scala/blob/master/src/compiler/scala/tools/nsc/settings/ScalaSettings.scala}
{scala/tools/nsc/settings/ScalaSettings.scala}}

\end{frame}

\begin{frame}[fragile]
\frametitle{Symbols}

Link definitions and references to definitions.
Long-lived, mutable.
Declared in \text{\color{linkblue}\href{https://github.com/scala/scala/blob/master/src/library/scala/reflect/api/Symbols.scala}
{scala/reflect/api/Symbols.scala}} (very much in flux!)

\begin{lstlisting}[language=Scala]
def foo[T: TypeTag](x: Any) = x.asInstanceOf[T]
foo[Long](42)
\end{lstlisting}

foo, T, x introduce symbols (T actually produces two different symbols).
DefDef, TypeDef, ValDef - all of those subtype DefTree.

TypeTag, x, T, foo, Long refer to symbols.
They are all represented by Idents, which subtype RefTree.

Symbols are long-lived. This means that any
reference to Int (from a tree or from a type) will point to the
same symbol instance.

\end{frame}

\begin{frame}[fragile]
\frametitle{Learn to learn}

\begin{itemize}
\item -Xprint:namer
\item -Yshow-trees and its cousins
\item -Yshow-syms and \text{\color{linkblue}\href{http://groups.google.com/group/scala-internals/browse_thread/thread/c068bc0ed331430f}
{much more}}
\item :type -v (see next slides)
\item -uniqid
\item kind and -Yshow-symkinds
\item Don't create them by yourself. Just don't, leave it to Namer.
\end{itemize}

\end{frame}

\begin{frame}[fragile]
\frametitle{Types}

Immutable, long-lived, sometimes cached case classes declared in \text{\color{linkblue}\href{https://github.com/scala/scala/blob/master/src/library/scala/reflect/api/Types.scala}{scala/reflect/api/Types.scala}}.

Store the information about the full wealth of the Scala type system:
members, type arguments, higher kinds, path dependencies, erasures, etc.

\end{frame}

\begin{frame}[fragile]
\frametitle{Learn to learn}

\begin{itemize}
\item -Xprint:typer
\item -Xprint-types
\item -explaintypes
\item :type -v
\item asSeenFrom
\end{itemize}

\end{frame}

\begin{frame}[fragile]
\frametitle{Learn to learn}

\begin{lstlisting}[language=XML]
scala> :type -v def impl[T: c.TypeTag](c: Context) = ???
// Type signature
[T](c: scala.reflect.makro.Context)(implicit evidence$1: c.TypeTag[T])Nothing

// Internal Type structure
PolyType(
  typeParams = List(TypeParam(T))
  resultType = MethodType(
    params = List(TermSymbol(c: ...))
    resultType = MethodType(
      params = List(TermSymbol(implicit evidence$1: ...))
      resultType = TypeRef(
        TypeSymbol(final abstract class Nothing)
      )
    )
  )
)
\end{lstlisting}
\end{frame}

\begin{frame}[fragile]
\frametitle{Big picture}

\begin{itemize}
\item Trees are created naked by Parser.
\item Both definitions and references (expressed as ASTs) get their symbols filled in by Namer (tree.symbol).
\item When creating symbols, Namer also creates their completers, lazy thunks that know how to populate symbol types (symbol.info).
\item Typer inspects trees, uses their symbols to transform trees and assign types to them (tree.tpe).
\item Shortly afterwards Pickler kicks in and serializes reachable symbols along with their types into ScalaSignature annotations.
\end{itemize}
\end{frame}

\section{Universes}

\begin{frame}[fragile]
\frametitle{Universes}

Universes are environments that pack together trees, symbols and their types.

\begin{itemize}
\item Compiler (scala.tools.nsc.Global) is a universe.
\item Reflective mirror (scala.reflect.mirror) is a universe too.
\item Macro context (scala.reflect.makro.Context) holds a reference to a universe.
\end{itemize}
\end{frame}

\begin{frame}[fragile]
\frametitle{Symbol tables}

Universes abstract population of symbol tables.

\begin{itemize}
\item Compiler loads symbols from pickles using its own *.class parser.
\item Reflective mirror uses Java reflection to load and parse ScalaSignatures.
\item Macro context refers to the compiler's symbol table.
\end{itemize}
\end{frame}

\begin{frame}[fragile]
\frametitle{Entry points}

Using a universe depends on your scenario.

\begin{itemize}
\item You can play with compiler's universe (aka global) in REPL's :power mode.
\item With mirrors you go through the \text{\color{linkblue}\href{https://github.com/scala/scala/blob/master/src/library/scala/reflect/api/Mirror.scala}{Mirror}}
interface, e.g. symbolOfInstance or classToType.
\item In a macro context, you import c.mirror.\_ and can use imported factories to create trees and types (don't create symbols manually, remember?).
\end{itemize}
\end{frame}

\begin{frame}[fragile]
\frametitle{Path dependency}

An important quirk is that all universe artifacts are path-dependent on their universe.

\begin{lstlisting}{language=Scala}
def reify[T](expr: T): Expr[T] = macro Universe.reify[T]
def reify[T]
  (cc: Context{ type PrefixType = Universe })
  (expr: cc.Expr[T])
  : cc.Expr[cc.prefix.value.Expr[T]] = ???
\end{lstlisting}

Hence to pass artifacts around outside the cake, you need to also carry the universe with you.
Or create your own cake using the technique outlined in
\text{\color{linkblue}\href{http://groups.google.com/group/scala-internals/msg/c861229693043723}{scala-internals discussion}}.
\end{frame}

\section{Demo}

\begin{frame}[fragile]
\frametitle{Reflection}

\begin{lstlisting}{language=XML}
scala> import scala.reflect.mirror
import scala.reflect.mirror

scala> trait X { def foo: String }
defined trait X

scala> typeTag[X]
res1: TypeTag[X] = ConcreteTypeTag[X]

scala> res1.tpe.members
res2: Iterable[reflect.mirror.Symbol] = List(method $asInstanceOf, method $isInstanceOf, method sync
hronized, method ##, method !=, method ==, method ne, method eq, constructor Object, method notifyAl
l, method notify, method clone, method getClass, method hashCode, method toString, method equals, me
thod wait, method wait, method wait, method finalize, method asInstanceOf, method isInstanceOf, meth
od !=, method ==, method foo)
\end{lstlisting}
\end{frame}

\begin{frame}[fragile]
\frametitle{Dynamic}

\begin{lstlisting}{language=Scala}
val d = new DynamicProxy{ val dynamicProxyTarget = x }
d.noargs
d.noargs()
d.nullary
d.value
d.-
d.$("x")
d.overloaded("overloaded with object")
d.overloaded(1)
val car = new Car
d.typeArgs(car) // inferred
d.default(1, 3)
d.default(1)
d.named(1, c=3, b=2) // works, wow
\end{lstlisting}

More examples at \text{\color{linkblue}\href{https://github.com/scala/scala/blob/master/test/files/run/dynamic-proxy.scala}{test/files/run/dynamic-proxy.scala}}.
Implementation is at \text{\color{linkblue}\href{https://github.com/scala/scala/blob/master/src/library/scala/reflect/DynamicProxy.scala}{scala/reflect/DynamicProxy.scala}}.

\end{frame}

\begin{frame}[fragile]
\frametitle{Runtime compilation}

\begin{lstlisting}{language=Scala}
import scala.reflect.mirror._
val tree = Apply(Select(Ident("Macros"), newTermName("foo")), List(Literal(Constant(42))))
println(Expr(tree).eval)
\end{lstlisting}

Eval creates a toolbox under the covers (universe.mkToolBox), which is a full-fledged compiler.

Toolbox wraps the input AST, sets its phase to Namer (skipping Parser) and performs the compilation into an in-memory directory.

After the compilation is finished, toolbox fires up a classloader that loads and lauches the code.

\end{frame}

\section{Macros}

\begin{frame}[fragile]
\frametitle{Macros}

In our hackings above, we used the runtime mirror (scala.reflect.mirror) to reflect against program structure.

We can do absolutely the same during the compile time.
The universe is already there (the compiler itself), the API is there as well
(scala.reflect.api.Universe inside the macro context).

We only need to ask the compiler to call ourselves during the compilation
(currently, our trigger is macro application and the hook is the macro keyword).

The end.

\end{frame}

\begin{frame}[fragile]
\frametitle{No, really}

That's it.

\end{frame}

\section{Summary}

\begin{frame}[fragile]
\frametitle{Summary}

\begin{itemize}
\item In 2.10 you can have all the information about your program that the compiler has (well, almost).
\item This information includes trees, symbols and types. And annotations. And positions.
\text{\color{linkblue}\href{https://github.com/scala/scala/blob/master/src/library/scala/reflect/api/Universe.scala}{And more}}.
\item You can reflect at runtime (scala.reflect.mirror) or at compile-time (macros).
\end{itemize}
\end{frame}

\begin{frame}[fragile]
\frametitle{Status}

We're already there. Nightlies already feature reflection and macros.
In a few days we will release 2.10.0 M3.

\end{frame}

% \begin{frame}[fragile]
% \frametitle{}
% \centerline{\includegraphics<1>[height=8cm]{scala-reflection.png}}
% \end{frame}

\begin{frame}[fragile]
\frametitle{Thanks!}

\centering
\text{\color{linkblue}\href{mailto:eugene.burmako@epfl.ch}{eugene.burmako@epfl.ch}}

\end{frame}

\end{document}
