\documentclass[hyperref={bookmarks=false}]{beamer}

\usepackage[parfill]{parskip}
\usepackage{color}
\usepackage{listings}
\usepackage{hyperref}
\hypersetup{pdfauthor={Eugene Burmako},pdfsubject={Metaprogramming in Scala 2.10},pdftitle={Metaprogramming in Scala 2.10}}
\title{Metaprogramming in Scala 2.10}

\definecolor{linkblue}{RGB}{49,57,174}
\definecolor{dkgreen}{rgb}{0,0.6,0}
\definecolor{gray}{rgb}{0.5,0.5,0.5}
\definecolor{mauve}{rgb}{0.58,0,0.82}

\lstdefinelanguage{scala}{
  morekeywords={abstract,annotation,case,catch,class,def,%
    do,else,extends,false,final,finally,%
    for,if,implicit,import,match,mixin,%
    new,null,object,override,package,%
    private,protected,requires,return,sealed,%
    super,this,throw,trait,true,try,%
    type,val,var,while,with,yield,
    macro},
  sensitive=true,
  morecomment=[l]{//},
  morecomment=[n]{/*}{*/}
%  morestring=[b]",
%  morestring=[b]',
%  morestring=[b]"""
}

\lstset{frame=tb,
  language=scala,
  aboveskip=3mm,
  belowskip=3mm,
  showstringspaces=false,
  columns=flexible,
  basicstyle={\small\ttfamily},
  numbers=none,
  numberstyle=\tiny\color{gray},
  keywordstyle=\color{blue},
  commentstyle=\color{dkgreen},
  stringstyle=\color{mauve},
  frame=single,
  breaklines=true,
  breakatwhitespace=true
  tabsize=3
}

\AtBeginSection[]
{
  \begin{frame}
    \frametitle{Agenda}
    \tableofcontents[currentsection]
  \end{frame}
}

\begin{document}

\title{Metaprogramming in Scala 2.10}
\author{Eugene Burmako}
\institute{EPFL, LAMP}
\date{28 April 2012 \\ (updated to accommodate changes in 2.10.0-M4 and M5)}
\maketitle

\section{Intro}

\begin{frame}[fragile]
\frametitle{Metaprogramming}

\begin{quote}
Metaprogramming is the writing of computer programs that write or manipulate other programs or themselves as their data.
\end{quote}

\begin{flushright}
\textemdash Wikipedia
\end{flushright}

\end{frame}

\begin{frame}[fragile]
\frametitle{Compiler}

Q: How to enable metaprogramming?

A: Who has more data about a program than a compiler?

Let's expose the compiler to the programmer.

\end{frame}

\begin{frame}[fragile]
\frametitle{Reflection}

In 2.10 we expose data about programs via reflection API.

The API is spread between scala-library.jar (interfaces), scala-reflect.jar (implementations) and scala-compiler.jar \\ (runtime compilation).

\end{frame}

\begin{frame}[fragile]
\frametitle{Martin knows better}

Design of the reflection API is explained in great detail in a recent talk by Martin Odersky:

\text{\color{linkblue}\href{http://channel9.msdn.com/Events/Lang-NEXT/Lang-NEXT-2012/Reflection-and-Compilers}
{channel9.msdn.com/Lang-NEXT-2012/Reflection-and-Compilers}}

\end{frame}

\begin{frame}[fragile]
\frametitle{Hands on}

Today we will learn the fundamentals of reflection API and learn to learn more about it
via a series of hands-on examples.

\end{frame}

\begin{frame}[fragile]
\frametitle{Macros}

Q: Hey! What about macros?

A: Reflection is at the core of macros, reflection provides macros with an API, reflection enables macros.
Our focus today is understanding reflection, macros are just a tiny bolt-on.

For more information about macros, their philosophy and applications, take a look at my Scala Days talk:
\text{\color{linkblue}\href{http://scalamacros.org/talks/2012-04-18-ScalaDays2012.pdf}
{http://scalamacros.org/talks/2012-04-18-ScalaDays2012.pdf}}.

\end{frame}

\section{Reflection}

\begin{frame}[fragile]
\frametitle{Core data structures}

\begin{itemize}
\item Trees
\item Symbols
\item Types
\end{itemize}

\begin{lstlisting}[language=XML]
C:\Projects\Kepler>scalac -Xshow-phases
phase name  id  description
----------  --  -----------
    parser   1  parse source into ASTs, simple desugaring
     namer   2  resolve names, attach symbols to named trees
     typer   4  the meat and potatoes: type the trees
   pickler   7  serialize symbol tables
\end{lstlisting}

I'll do my best to explain these concepts, but it's barely possible to do it better than Paul Phillips.
Be absolutely sure to watch the \text{\color{linkblue}\href{http://skillsmatter.com/podcast/scala/scalac-internals}
{Inside the Sausage Factory}} talk (when the video is up).
\end{frame}

\begin{frame}[fragile]
\frametitle{Trees}

Short-lived, mostly immutable, mostly plain case classes.

\begin{lstlisting}[language=Scala]
Apply(Ident("println"), List(Literal(Constant("hi!"))))
\end{lstlisting}

Comprehensive list of public trees can be found here:
\text{\color{linkblue}\href{https://github.com/scala/scala/blob/master/src/library/scala/reflect/base/Trees.scala}{scala/reflect/base/Trees.scala}}. Also take a look at the cousin of the linked file at \text{\color{linkblue}\href{https://github.com/scala/scala/blob/master/src/reflect/scala/reflect/api/Trees.scala}{scala/reflect/api/Trees.scala}}.

Why the split between base and api? The former is available everywhere, the latter requires scala-reflect.jar.

\end{frame}

\begin{frame}[fragile]
\frametitle{Learn to learn}

\begin{itemize}
\item -Xprint:parser (for naked trees)
\item -Xprint:typer (for typechecked trees)
\item -Yshow-trees and its cousins
\item ru.showRaw(ru.reify(...)) // where ru stands for scala.reflect.runtime.universe
\item also check out the optional parameters of showRaw!
\end{itemize}

Q: Where do I pull these compiler flags from?

A: \text{\color{linkblue}\href{https://github.com/scala/scala/blob/master/src/compiler/scala/tools/nsc/settings/ScalaSettings.scala}
{scala/tools/nsc/settings/ScalaSettings.scala}}

\end{frame}

\begin{frame}[fragile]
\frametitle{-Yshow-trees}

\begin{lstlisting}[language=XML]
// also try -Yshow-trees-stringified
// and -Yshow-trees-compact (or both simultaneously)
>scalac -Xprint:parser -Yshow-trees HelloWorld.scala
[[syntax trees at end of parser]]// Scala source: HelloWorld.scala
PackageDef(
  "<empty>"
  ModuleDef(
    0
    "Test"
    Template(
      "App" // parents
      ValDef(
        private
        "_"
        <tpt>
        <empty>
      )
      ...
\end{lstlisting}

\end{frame}

\begin{frame}[fragile]
\frametitle{showRaw}

\begin{lstlisting}[language=XML]
// ru stands for scala.reflect.runtime.universe
scala> ru.reify{ object Test { println("Hello World!") } }
res0: reflect.runtime.universe.Expr[Unit] = ...

scala> ru.showRaw(res0.tree)
res1: String = Block(List(ModuleDef(
  Modifiers(),
  newTermName("Test"),
  Template(List(Ident(newTypeName("AnyRef"))), List(
    DefDef(Modifiers(), newTermName("<init>"), List(), List(List()), TypeTree(), Block(List(Apply(Select(Super(This(newTypeName("")), newTypeName("")), newTermName("<init>")), List())), Literal(Constant(())))),
    Apply(Select(Select(This(newTypeName("scala")), newTermName("Predef")), newTermName("println")), List(Literal(Constant("Hello World!")))))))),
Literal(Constant(())))
\end{lstlisting}

\end{frame}

\begin{frame}[fragile]
\frametitle{Symbols}

Link definitions and references to definitions.
Long-lived, mutable.
Declared in \text{\color{linkblue}\href{https://github.com/scala/scala/blob/master/src/library/scala/reflect/base/Symbols.scala}
{scala/reflect/base/Symbols.scala}} and \text{\color{linkblue}\href{https://github.com/scala/scala/blob/master/src/reflect/scala/reflect/api/Symbols.scala}
{scala/reflect/api/Symbols.scala}}

\begin{lstlisting}[language=Scala]
def foo[T: TypeTag](x: Any) = x.asInstanceOf[T]
foo[Long](42)
\end{lstlisting}

foo, T, x introduce symbols (T actually produces two different symbols, but that's a different story).
DefDef, TypeDef, ValDef - all of those subtype DefTree.

TypeTag, x, T, foo, Long refer to symbols.
They are all represented by Idents, which subtype RefTree.

Symbols are long-lived. This means that any
reference to Int (from a tree or from a type) will point to the
same symbol instance.

\end{frame}

\begin{frame}[fragile]
\frametitle{Learn to learn}

\begin{itemize}
\item -Xprint:namer or -Xprint:typer
\item -uniqid
\item symbol.kind and -Yshow-symkinds
\item :type -v
\item showRaw(tree, printIds = true, printKinds = true)
\item Don't create them by yourself. Just don't, leave it to Namer. In macros most of the time you create naked trees, and Typer will take care of the rest. Sometimes it inevitable, though: \text{\color{linkblue}\href{http://stackoverflow.com/questions/11208790/how-can-i-reuse-definition-ast-subtrees-in-a-macro}{http://stackoverflow.com/questions/11208790}}.
\end{itemize}

\end{frame}

\begin{frame}[fragile]
\frametitle{-uniqid and -Yshow-symkinds}

\begin{lstlisting}[language=XML]
>cat Foo.scala
def foo[T: TypeTag](x: Any) = x.asInstanceOf[T]
foo[Long](42)

// there is a mysterious factoid hidden in this printout!
>scalac -Xprint:typer -uniqid -Yshow-symkinds Foo.scala
[[syntax trees at end of typer]]// Scala source: Foo.scala
def foo#8339#METH
  [T#8340#TPE >: Nothing#4658#CLS <: Any#4657#CLS]
  (x#9529#VAL: Any#4657#CLS)
  (implicit evidence$1#9530#VAL: TypeTag#7861#TPE[T#8341#TPE#SKO])
  : T#8340#TPE =
  x#9529#VAL.asInstanceOf#6023#METH[T#8341#TPE#SKO];

Test#14#MODC.this.foo#8339#METH[Long#1641#CLS](42)
(scala#29#PK.reflect#2514#PK.`package`#3414#PKO
.mirror#3463#GET.TypeTag#10351#MOD.Long#10361#GET)
\end{lstlisting}

\end{frame}

\begin{frame}[fragile]
\frametitle{:type -v}

We have just seen how to discover symbols used in trees.

However, symbols are also used in types.

Thanks to Paul (who hacked this during one of Scala Nights)
there's an easy way to inspect types as well.
Corresponding REPL incantation is shown on one of the next slides.

Starting from 2.10.0-M5 you can also use showRaw (defined in all universes:
the scala.reflect.runtime.universe, all macro context universes)
to print out raw structure of types.

\end{frame}

\begin{frame}[fragile]
\frametitle{Types}

Immutable, long-lived, sometimes cached case classes declared in \text{\color{linkblue}\href{https://github.com/scala/scala/blob/master/src/library/scala/reflect/base/Types.scala}{scala/reflect/base/Types.scala}} and \text{\color{linkblue}\href{https://github.com/scala/scala/blob/master/src/reflect/scala/reflect/api/Types.scala}{scala/reflect/api/Types.scala}}.

Store the information about the full wealth of the Scala type system:
members, type arguments, higher kinds, path dependencies, erasures, etc.

\end{frame}

\begin{frame}[fragile]
\frametitle{Learn to learn}

\begin{itemize}
\item -Xprint:typer
\item -Xprint-types
\item :type -v
\item showRaw(type, printIds = true, printKinds = true)
\item -explaintypes
\end{itemize}

\end{frame}

\begin{frame}[fragile]
\frametitle{-Xprint-types}

-Xprint-types is yet another option that modifies tree printing.
Nothing very fancy, let's move on to something really cool.

\end{frame}

\begin{frame}[fragile]
\frametitle{:type -v}

\begin{lstlisting}[language=XML]
scala> :type -v def impl[T: c.TypeTag](c: Context) = ???
// Type signature
[T](c: scala.reflect.makro.Context)(implicit evidence$1: c.TypeTag[T])Nothing

// Internal Type structure
PolyType(
  typeParams = List(TypeParam(T))
  resultType = MethodType(
    params = List(TermSymbol(c: ...))
    resultType = MethodType(
      params = List(TermSymbol(implicit evidence$1: ...))
      resultType = TypeRef(
        TypeSymbol(final abstract class Nothing)
      )
    )
  )
)
\end{lstlisting}
\end{frame}

\begin{frame}[fragile]
\frametitle{showRaw (available since 2.10.0-M5)}

\begin{lstlisting}[language=XML]
scala> object O {
  def impl[T: c.TypeTag](c: Context) = ???
}
defined module O

scala> val meth = ru.reify(O).staticTpe.typeSymbol.
  typeSignature.member(newTermName("impl"))
meth: reflect.runtime.universe.Symbol = method impl

scala> println(showRaw(meth.typeSignature))
PolyType(
  List(newTypeName("T")),
  MethodType(List(newTermName("c")),
    MethodType(List(newTermName("evidence$1")),
      TypeRef(ThisType(scala), scala.Nothing, List()))))
\end{lstlisting}
\end{frame}

\begin{frame}[fragile]
\frametitle{-explaintypes}

\begin{lstlisting}[language=XML]
>cat Test.scala
class Foo { class Bar; def bar(x: Bar) = ??? }

object Test extends App {
  val foo1 = new Foo
  val foo2 = new Foo
  foo2.bar(new foo1.Bar)
}

// prints explanations of type mismatches
>scalac -explaintypes Test.scala
Test.foo1.Bar <: Test.foo2.Bar?
  Test.foo1.type <: Test.foo2.type?
    Test.foo1.type = Test.foo2.type?
    false
  false
false
Test.scala:6: error: type mismatch;
...
\end{lstlisting}
\end{frame}

\begin{frame}[fragile]
\frametitle{Big picture}

\begin{itemize}
\item Trees are created naked by Parser.
\item Both definitions and references (expressed as ASTs) get their symbols filled in by Namer (tree.symbol).
\item When creating symbols, Namer also creates their completers, lazy thunks that know how to populate symbol types (symbol.info).
\item Typer inspects trees, uses their symbols to transform trees and assign types to them (tree.tpe).
\item Shortly afterwards Pickler kicks in and serializes reachable symbols along with their types into ScalaSignature annotations.
\end{itemize}
\end{frame}

\section{Universes}

\begin{frame}[fragile]
\frametitle{Universes}

Universes are environments that pack together trees, symbols and their types.

\begin{itemize}
\item Compiler (scala.tools.nsc.Global) is a universe.
\item Reflection runtime (scala.reflect.runtime.universe) is a universe too.
\item Macro context (scala.reflect.makro.Context) holds a reference to a universe.
\end{itemize}
\end{frame}

\begin{frame}[fragile]
\frametitle{Mirrors}

Mirrors abstract population of symbol tables.

Each universe can have multiple mirrors, which can share symbols within their parent universe. \text{\color{linkblue}\href{https://docs.google.com/document/d/1Z1VhhNPplbUpaZPIYdc0_EUv5RiGQ2X4oqp0i-vz1qw/edit}{Scala Reflection writeup}} elaborates on this in more detail.

\begin{itemize}
\item Compiler loads symbols from pickles using its own *.class parser. It has only one mirror, the rootMirror.
\item Reflective mirror uses Java reflection to load and parse ScalaSignatures. Every classloader corresponds to its own mirror created with ru.runtimeMirror(classloader).
\item Macro context refers to the compiler's symbol table.
\end{itemize}
\end{frame}

\begin{frame}[fragile]
\frametitle{Entry points}

Using a universe depends on your scenario.

\begin{itemize}
\item You can play with compiler's universe (aka global) in REPL's :power mode.
\item With runtime reflection you typically go through the \text{\color{linkblue}\href{https://github.com/scala/scala/blob/master/src/reflect/scala/reflect/api/Mirror.scala}{Mirror}}
interface, e.g. scala.reflect.runtime.currentMirror, then cm.reflect and then you can get/set fields, invoke methods, etc. Read up more on \text{\color{linkblue}\href{http://stackoverflow.com/questions/11020746/get-companion-object-instance-with-new-scala-reflection-api}{stackoverflow}}.
\item In a macro context, you import c.universe.\_ and can use imported factories to create trees and types (don't create symbols manually, remember?).
\end{itemize}
\end{frame}

\begin{frame}[fragile]
\frametitle{Path dependency}

An important quirk is that all universe artifacts are path-dependent on their universe.
Note the reflect.runtime.universe prefix in the type of the result printed below.

\begin{lstlisting}{language=XML}
scala> ru.reify(2.toString)
res0: reflect.runtime.universe.Expr[String] = Expr[String](2.toString())
\end{lstlisting}

When you deal with runtime reflection, you simply import scala.reflect.runtime.universe.\_, and enjoy,
because typically there is only one runtime universe.

However with macros it's more complicated. To pass artifacts around (e.g. into helper functions),
you need to also carry the universe with you. Or you can employ the technique outlined in
\text{\color{linkblue}\href{http://groups.google.com/group/scala-internals/msg/c861229693043723}{a discussion at scala-internals}}.
\end{frame}

\section{Demo}

\begin{frame}[fragile]
\frametitle{Inspect members}

\begin{lstlisting}{language=XML}
scala> import scala.reflect.runtime.{universe => ru}
import scala.reflect.runtime.{universe=>ru}

scala> trait X { def foo: String }
defined trait X

scala> ru.typeOf[X]
res0: reflect.runtime.universe.Type = X

scala> res0.members
res1: Iterable[reflect.runtime.universe.Symbol] = List(method $asInstanceOf, method $isIn
stanceOf, method synchronized, method ##, method !=, method ==, method ne, method eq, constructor Ob
ject, method notifyAll, method notify, method clone, method getClass, method hashCode, method toStri
ng, method equals, method wait, method wait, method wait, method finalize, method asInstanceOf, meth
od isInstanceOf, method !=, method ==, method foo)
\end{lstlisting}
\end{frame}

\begin{frame}[fragile]
\frametitle{Analyze and invoke members}

Daniel Sobral got it all covered in his series of articles about reflection:
\text{\color{linkblue}\href{http://dcsobral.blogspot.ch/2012/07/json-serialization-with-reflection-in.html}{JSON serialization with reflection in Scala!}}

\end{frame}

\begin{frame}[fragile]
\frametitle{Defeat erasure}

\begin{lstlisting}{language=XML}
scala> def foo[T](x: T) = x.getClass
foo: [T](x: T)Class[_ <: T]

scala> foo(List(1, 2, 3))
res0: Class[_ <: List[Int]] = class scala.collection.immutable.$colon$colon

scala> def foo[T: ru.TypeTag](x: T) = ru.typeOf[T]
foo: [T](x: T)(implicit evidence$1: ru.TypeTag[T])ru.Type

scala> foo(List(1, 2, 3))
res1: reflect.runtime.universe.Type = List[Int]
\end{lstlisting}
\end{frame}

\begin{frame}[fragile]
\frametitle{Compile at runtime}

\begin{lstlisting}{language=Scala}
import scala.reflect.runtime.universe._
import scala.tools.reflect.ToolBox
val tree = Apply(Select(Literal(Constant(40)), newTermName("$plus")), List(Literal(Constant(2))))
val cm = ru.runtimeMirror(getClass.getClassLoader)
println(cm.mkToolBox().runExpr(tree))
\end{lstlisting}

Toolbox is a full-fledged compiler (the scala.tools.reflect.ToolBox import requires scala-compiler.jar on the classpath). Unlike the regular compiler, it uses Java reflection encapsulated in the provided mirror to populate its symbol table.

Toolbox wraps the input AST, sets its phase to Namer (skipping Parser) and performs the compilation into an in-memory directory.

After the compilation is finished, toolbox fires up a classloader that loads and lauches the code.

\end{frame}

\section{Macros}

\begin{frame}[fragile]
\frametitle{Macros}

In our hackings above, we used the runtime universe (scala.reflect.runtime.universe) to reflect against program structure.

We can do absolutely the same during the compile time.
The universe is already there (the compiler itself), the API is there as well
(scala.reflect.api.Universe inside the macro context).

We only need to ask the compiler to call ourselves during the compilation
(currently, our trigger is macro application and the hook is the macro keyword).

The end.

\end{frame}

\begin{frame}[fragile]
\frametitle{No, really}

That's it.

\end{frame}

\section{Summary}

\begin{frame}[fragile]
\frametitle{Summary}

\begin{itemize}
\item In 2.10 you can have all the information about your program that the compiler has (well, almost).
\item This information includes trees, symbols and types. And annotations. And positions.
\text{\color{linkblue}\href{https://github.com/scala/scala/blob/master/src/reflect/scala/reflect/api/Universe.scala}{And more}}.
\item You can reflect at runtime (scala.reflect.runtime.universe) or at compile-time (macros).
\end{itemize}
\end{frame}

\begin{frame}[fragile]
\frametitle{Status}

We're already there.

Since 2.10.0-M3 Scala includes reflection and macros.

\end{frame}

% \begin{frame}[fragile]
% \frametitle{}
% \centerline{\includegraphics<1>[height=8cm]{scala-reflection.png}}
% \end{frame}

\begin{frame}[fragile]
\frametitle{Thanks!}

\centering
\text{\color{linkblue}\href{mailto:eugene.burmako@epfl.ch}{eugene.burmako@epfl.ch}}

\end{frame}

\end{document}
